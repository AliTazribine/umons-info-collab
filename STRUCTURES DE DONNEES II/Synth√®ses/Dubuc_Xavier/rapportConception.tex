\documentclass{article}

\usepackage{amsmath}
\usepackage{amsfonts}
\usepackage{amssymb}
\usepackage{multicol}

\usepackage{mathenv}

\def\nbOne{{\mathchoice {\rm 1\mskip-4mu l} {\rm 1\mskip-4mu l}
{\rm 1\mskip-4.5mu l} {\rm 1\mskip-5mu l}}}

\usepackage{vmargin}
\setmarginsrb{2.5cm}{2.5cm}{2.5cm}{2.5cm}{0cm}{0cm}{0cm}{0cm}

\usepackage[utf8]{inputenc}
\usepackage{algorithm,algorithmic}

\usepackage[french]{babel}
\selectlanguage{french}

\usepackage{color}
\usepackage{graphicx}
\graphicspath{{img/}} 
\usepackage{listings}
\definecolor{colKeys}{rgb}{0.75,0,0}
\definecolor{colIdentifier}{rgb}{0,0,0}
\definecolor{colComments}{rgb}{0.75,0.75,0}
\definecolor{colString}{rgb}{0,0,0.7}

\lstset{
basicstyle=\ttfamily\small, %
identifierstyle=\color{colIdentifier}, %
keywordstyle=\color{colKeys}, %
stringstyle=\color{colString}, %
commentstyle=\color{colComments}, %
showspaces=false,
}
\lstset{language=java}

% Commandes personnelles %

\definecolor{darkred}{rgb}{0.85,0,0}
\definecolor{darkblue}{rgb}{0,0,0.7}
\definecolor{darkgreen}{rgb}{0,0.6,0}
\definecolor{darko}{rgb}{0.93,0.43,0}
\newcommand{\dred}[1]{\textcolor{darkred}{\textbf{#1}}}
\newcommand{\dgre}[1]{\textcolor{darkgreen}{\textbf{#1}}}
\newcommand{\dblu}[1]{\textcolor{darkblue}{\textbf{#1}}}
\newcommand{\dora}[1]{\textcolor{darko}{\textbf{#1}}}
\newcommand{\gre}[1]{\textcolor{darkgreen}{#1}}
\newcommand{\blu}[1]{\textcolor{darkblue}{#1}}
\newcommand{\ora}[1]{\textcolor{darko}{#1}}
\newcommand{\red}[1]{\textcolor{darkred}{#1}}
\newcommand{\ceil}[1]{\left\lceil #1 \right\rceil}
\newcommand{\cdil}[1]{\left\lfloor #1 \right\rfloor}
\newcommand{\image}[1]{\includegraphics{#1}}
\newcommand{\imageR}[2]{\includegraphics[width=#2px]{#1}}
\newcommand{\imageRT}[2]{\includegraphics[height=#2px]{#1}}
\newcommand{\img}[1]{\begin{center}\includegraphics[width=400px]{#1}\end{center}}
\newcommand{\imag}[1]{\begin{center}\includegraphics{#1}\end{center}}
\newcommand{\imgR}[2]{\begin{center}\includegraphics[width=#2px]{#1}\end{center}}
\newcommand{\imgRT}[2]{\begin{center}\includegraphics[height=#2px]{#1}\end{center}}
\newcommand{\point}[2]{\item \ora{\underline{#1}} : \textit{#2}}
\newcommand{\bfp}[2]{\item \textbf{#1} : \textit{#2}}
\newcommand{\sumparam}[3]{\sideset{}{_{#1}^{#2}}\sum{#3}}
\newcommand{\sumin}[3]{\sideset{}{_{i=#1}^{#2}}\sum{#3}}
\newcommand{\sumkn}[3]{\sideset{}{_{k=#1}^{#2}}\sum{#3}}
\newcommand{\intin}[3]{\sideset{}{_{#1}^{#2}}\int{#3}}
\newcommand{\stitre}[1]{\noindent\textbf{\underline{#1}} \\}
\newcommand{\R}{\mathbb{R}}
\newcommand{\Z}{\mathbb{Z}}
\newcommand{\N}{\mathbb{N}}
\DeclareMathAlphabet{\mathpzc}{OT1}{pzc}{m}{it}

\title{\textbf{\textcolor{darkblue}{Structures de Données - Rapport de conception.}}}
\author{\textit{Dubuc Xavier \& Hannecart Aurore}}

\begin{document}

\maketitle

\hbox{\raisebox{0.4em}{\vrule depth 0.4pt height 0.4pt width 10cm}}

\tableofcontents

$ $ \\
\hbox{\raisebox{0.4em}{\vrule depth 0.4pt height 0.4pt width 10cm}}

\newpage

\section{Introduction}

Dans le cadre du cours de \textit{Structures de données II}, nous avons été amenés à mener à bien un 
projet concernant les arbres \textbf{BSP}\textit{(Binary Space Partition)}. Ces arbres sont des 
arbres binaires de recherche spéciaux permettant de partitionner une scène constituée de segments. 
Notre projet consiste à afficher ce que voit un \oe il d'une scène en 2 dimensions sur une droite en 
utilisant l'\textbf{algorithme du peintre}. Pour ce faire, nous devons construire un arbre 
\textbf{BSP} selon une heuristique fournie et interroger celui-ci.

\section{Classes utilisées}

\subsection{Couleur.java}

Cette classe est une classe plus utile que nécessaire, en effet, elle permet simplement de faire la 
correspondance entre une chaîne de caractères et un objet \textit{Color}. Ainsi, on crée des objets 
contenant par exemple «Bleu» et \textit{Color.\blu{BLUE}} et on implémente des méthodes utiles comme 
\textit{equals()} et \textit{toString()} afin de faciliter la correspondance décrite ci-dessus.

\subsection{Palette.java}

Cette classe \textit{abstraite} représente simplement la palette de couleurs que l'on utilise dans 
ce projet. Elle contient un tableau d'objet \textit{Couleur} et implémente quelques méthodes de 
classe permettant d'obtenir l'objet \textit{Couleur}, l'objet \textit{Color} ou la chaine de 
caractères caractérisant une couleur en fonction d'une chaine de caractères d'un objet 
\textit{Couleur} ou d'un objet \textit{Color}.

\subsection{EquationDroite.java}

Cette interface permet de symboliser l'équation d'une droite, elle demande comme méthode 
\textit{eval(double x, double y)}. Elle nous servira pour calculer $ax+by+c$ pour un point désiré.

\subsection{Segment.java}

Cette classe permet de symboliser un segment, un segment étant caractérisé par 2 points 
(classe java \textit{Point2D.Double}), une couleur (classe \textit{Couleur}) et l'équation de la 
droite qui le porte (interface \textit{EquationDroite}). On y implémente la méthode 
\textit{equals()} de manière à ce que 2 segments soient égaux si ils sont de même couleur et ont les 
2 mêmes points les définissant. On implémente également une méthode \textit{hasIntersectionWith()} 
ainsi que \textit{getIntersectionPointWith()} qui prennent toutes 2 une \textbf{EquationDroite} en 
paramètre. La première permet de savoir si la droite portant le segment possède un et un seul point 
d'intersection avec la droite passée en paramètre et la seconde permet de calculer ce point.

\subsection{SceneReader.java}

Cette classe permet de lire un fichier \textit{.txt} représentant une scène et de convertir les 
informations en un Arraylist de segments tout en conservant également le nombre de segments, 
l'abscisse maximale et l'ordonnée maximale.

\subsection{BSPTree.java}

Cette classe représente la structure de données des arbres \textbf{BSP}, nous avons décidé 
d'utiliser une seule classe, classe pouvant représenter une feuille contenant un seul segment, vide 
ou un noeud ayant 1 ou 2 fils. Elle possède un ArrayList contenant tous les segments qui sont 
contenus dans le séparateur, séparateur représenté par une \textbf{EquationDroite}. Elle possède 
également 2 booléens spécifiant si le noeud est une feuille et si celui-ci est vide ainsi que 2 
\textbf{BSPTree} symbolisant les 2 fils du noeud. (Ces 2 fils pouvant ne pas exister)

\subsection{Heuristic.java}

Interface permettant de définir un heuristique. Elle demande la méthode 
\textit{getDroite(ArrayList$<$Segments$>$ a)}, méthode qui est appellée à chaque séparation 
effectuée. Celle-ci permet donc de définir quelle droite utiliser pour séparer la scène. Comme 
exemple simple, l'heuristique consistant à choisir la droite portant le premier segment de 
l'ensemble consiste à simplement retourner $a.get(0)$.

\subsection{BSPTreeBuilder.java}

Cette classe \textit{abstraite} permet de construire un arbre \textbf{BSP} en fonction d'une 
heuristique prédéfinie. Elle consiste en l'application de l'algorithme vu dans le livre en 
permettant de spécifier l'heuristique à utiliser, cette classe est donc totalement générale et 
pourra être utilisée pour toutes les heuristiques que l'on aura à implémenter.

\subsection{Vue.java}

Définie par 2 points, un point de situation et un point d'atteinte. Le premier est le point où se 
situe l'\oe il et le second permet de définir dans quel sens regarde cet \oe il.

\subsection{DrawPanel.java}

Classe étendant la classe \textbf{JPanel} de java et qui permet de dessiner le segment vu par un
\oe il défini.

\subsection{Painter.java}

Cette classe abstraite permet d'appliquer l'algorithme du peintre et d'afficher la réponse demandée 
par l'énoncé, c'est-à-dire le segment représentant ce que voit la \textbf{Vue} via un 
\textbf{DrawPanel}. Elle possède 3 méthodes importantes, 
\begin{itemize}
\item \textit{getDepthOrder(BSPTree \red{tree}, Vue \red{v})} qui renvoie un ensemble de segments 
triés par ordre d'apparition inversé face à l'oeil (le dernier segment de la liste est le premier 
devant l'\oe il)
\item \textit{getProjectedSegment(Vue \red{v}, Segment \red{a})} qui renvoie un \textbf{Segment} 
correspondant à la projection de \red{a} sur la droite perpendiculaire au vecteur définissant l'\oe 
il.
\item \textit{getViewedSegment(Vue \red{v}, ArrayList$<$Segment$>$ \red{a})} qui appelle la méthode 
précédente pour chaque segment de l'ensemble de segments triés par ordre d'apparition afin de 
dessiner une droite représentant ce que voit l'oeil. Il reste ensuite à convertir les coordonnées 
afin que la droite soit horizontale et bien centrée pour l'afficher sans problème.
\end{itemize}
Cette manière de faire n'est pas très optimale car on parcoure 3 fois l'arbre ou la liste des 
segments pour effectuer un travail que l'on peut faire en parcourant une seule fois l'arbre (en se 
collant complètement à l'algorithme explicité dans le livre). Nous avons cependant décidé dans un 
premier temps de tout séparer afin que le code soit plus clair et plus compréhensible pour nous, 
lors de l'implémentation, le code sera optimisé.


\section{Algorithmes}

\subsection{Algorithme 1 : Création d'un arbre BSP}

\begin{algorithm}
\caption{BuildBSPTree(S,h)}
\textbf{Entrée} : \textit{\red{S}, une liste de segment de taille $n$ représentant une scène, 
\red{h}, l'heuristique à suivre pour construire l'arbre.} \\
\textbf{Sortie} : \textit{\red{T}, l'arbre BSP correspondant.}
\begin{algorithmic}
\IF{($n=0$)}
\STATE $T_{separator} \leftarrow $ vide
\STATE $T_{left} \leftarrow $ vide
\STATE $T_{right} \leftarrow $ vide
\STATE $T_{segments} \leftarrow $ vide
\STATE $T_{isLeaf} \leftarrow $ vrai
\STATE $T_{isEmpty} \leftarrow $ vrai
\ELSE
	\IF{($n=1$)}
		\STATE $T_{separator} \leftarrow \red{S}[0]_{equation}$
		\STATE $T_{left} \leftarrow$ vide
		\STATE $T_{right} \leftarrow$ vide
		\STATE $T_{isLeaf} \leftarrow$ vrai
		\STATE $T_{isEmpty} \leftarrow$ faux
		\STATE $Ajout(\red{S}[0],T_{segments})$
	\ELSE
		\STATE $T_{separator} \leftarrow$ \textit{GetDroite(\red{S},\red{h})}
		\STATE $T_{isLeaf} \leftarrow$ faux
		\STATE $T_{isEmpty} \leftarrow$ faux
		\STATE $S^+ \leftarrow nouvelleListe()$
		\STATE $S^- \leftarrow nouvelleListe()$
		\FOR{$i$ allant de $0$ à $n-1$}
		\STATE s $\leftarrow$ \red{S}[i]
		\STATE $valBegin \leftarrow Evaluation(T_{separator}, s_{begin})$
		\STATE $valEnd \leftarrow Evaluation(T_{separator}, s_{end})$
		\IF{($valBegin = 0$ ET $valEnd = 0$)}
			\STATE $Ajout(s,T_{segment})$
			\IF{($valBegin \leq 0$ ET $valEnd \leq 0$)}
				\STATE $Ajout(s,S^-)$
			\ELSE
				\IF{($valBegin \geq 0$ ET $valEnd \geq 0$)}
					\STATE $Ajout(s,S^+)$
				\ELSE
					\STATE $m \leftarrow Intersection(T_{separator},s)$
					\STATE $s_1 \leftarrow (s_{begin}, m, s_{color})$
					\STATE $s_2 \leftarrow (m,s_{end}, s_{color})$
					\IF{($valBegin \leq 0$)}
						\STATE $Ajout(s_1,S^-)$
						\STATE $Ajout(s_2,S^+)$
					\ELSE
						\STATE $Ajout(s_1,S^+)$
						\STATE $Ajout(s_2,S^-)$
					\ENDIF
				\ENDIF
			\ENDIF
		\ENDIF
		\ENDFOR
		\STATE $T_{left} \leftarrow$ \textbf{BuildBSPTree$(S^-,h)$}
		\STATE $T_{right} \leftarrow$ \textbf{BuildBSPTree$(S^+,h)$}
		\STATE \textbf{retourner} T
	\ENDIF
\ENDIF
\end{algorithmic}
\end{algorithm}

\subsection{Algorithme 2 : Heuristique}

\noindent \underline{Rappel de l'heuristique} : heuristique numéro $4$, l'heuristique de 
\textit{Teller} ; l'idée est de choisir la droite $d$ qui :
\begin{itemize}
\item \textbf{maximise} $\sigma_d$ \textit{(la proportion de segments coupés par $d$)} si $\sigma_d$ 
est $\geq$ qu'un certain seuil donné représenté par un nombre réel,
\item \textbf{minimise} $f_d$ \textit{(le nombre de segments coupés par $d$)} sinon.
\end{itemize}

\begin{algorithm}
\caption{GetDroite($I_v$,$\tau$)}
\textbf{Entrée} : \textit{\red{$I_v$}, une liste de segment de taille $n$ représentant une scène, 
\red{$\tau$} un nombre réel représentant le seuil.} \\
\textbf{Sortie} : \textit{\red{d}, l'équation de la droite à utiliser comme séparateur.}
\begin{algorithmic}
\STATE $(max_{\sigma_d}, segmentMax_{\sigma_d}) \leftarrow (0,(+\infty,+\infty))$
\STATE $(min_{F_d}, segmentMin_{F_d}) \leftarrow (+\infty,(+\infty,+\infty))$
\FORALL{les segments $s$ de $I_v$}
	\STATE $d \leftarrow$ \textit{droite supportant s}
	\STATE $f_d \leftarrow 0$
	\STATE $\sigma_d \leftarrow 0$
	\FORALL{les segments $s'$ de $I_v$ (sauf s)}
		\IF{\textit{d intersecte s'}}
			\STATE $f_d \leftarrow f_d + 1$
		\ENDIF
	\ENDFOR
	\STATE $\sigma_d \leftarrow \dfrac{f_d}{|I_v|}$
	\IF{$\sigma_d \geq max_{\sigma_d}$}
		\STATE $(max_{\sigma_d}, segmentMax_{\sigma_d}) \leftarrow (\sigma_d,s)$
	\ENDIF
	\IF{$f_d \geq min_{f_d}$}
		\STATE $(min_{f_d}, segmentMin_{f_d}) \leftarrow (f_d,s)$
	\ENDIF
\ENDFOR
\IF{$\sigma_d \geq \red{\tau}$} 
	\STATE \textit{\textbf{retourner} la droite portant $segmentMax_{\sigma_d}$}
\ELSE
	\STATE \textit{\textbf{retourner} la droite portant $segmentMin_{f_d}$}
\ENDIF
\end{algorithmic}
\end{algorithm}

\subsection{Algorithme 3 : algorithme du peintre}

L'idée ici est de parcourir l'arbre selon un ordre défini et de travailler sur chaque segment 
considéré. Le travail consiste à projeter le segment sur une droite derrière la scène, de vérifier 
s'il n'est pas caché et de le dessiner sur la droite (avec sa couleur). Comme nous avons séparé le 
travail, l'algorithme du peintre tel que nous l'avons utilisé est très léger et simple, il nous 
permet uniquement de récupérer les segments dans l'ordre dans lequel l'\oe il les voit. Voici donc 
cet algorithme, qui est pratiquement identique à celui du livre.

\begin{algorithm}
\caption{GetDepthOrder($T$,$v$)}
\textbf{Entrée} : \textit{\red{$T$}, un arbre BSP, $v$ une vue (telle que décrite plus haut).}\\
\textbf{Sortie} : \textit{\red{$A$}, un arraylist contenant les segments dans l'ordre décrit ci-
haut.}
\begin{algorithmic}
\IF{\textit{T est vide}}
\STATE \textit{\textbf{retourner} arraylist vide}
\ELSE
	\IF{\textit{T est une feuille}}
		\STATE $Ajout(\red{A}, T_{segments}[0])$
		\STATE \textit{\textbf{retourner} $\red{A}$}
	\ELSE
		\STATE $\red{A} \leftarrow$ \textit{arraylist vide}
		\STATE $d \leftarrow T_{separator}$
		\STATE $val \leftarrow Evaluation(d,v)$
		\STATE $Right \leftarrow$ \textbf{GetDepthOrder}$(T_{right},v)$
		\STATE $Left \leftarrow$ \textbf{GetDepthOrder}$(T_{left},v)$
		\IF{$val = 0$}
		\FORALL{\textit{segments \textbf{s} dans $Right$}}
			\STATE $Ajout(\red{A},s)$
		\ENDFOR
		\FORALL{\textit{segments \textbf{s} dans $Left$}}
			\STATE $Ajout(\red{A},s)$
		\ENDFOR
		\ELSE
			\IF{$val \leq 0$}
				\FORALL{\textit{segments \textbf{s} dans $Right$}}
					\STATE $Ajout(\red{A},s)$
				\ENDFOR
				\FORALL{\textit{segments \textbf{s} dans $T_{segment}$}}
					\STATE $Ajout(\red{A},s)$
				\ENDFOR				
				\FORALL{\textit{segments \textbf{s} dans $Left$}}
					\STATE $Ajout(\red{A},s)$
				\ENDFOR
			\ELSE
				\FORALL{\textit{segments \textbf{s} dans $Left$}}
					\STATE $Ajout(\red{A},s)$
				\ENDFOR
				\FORALL{\textit{segments \textbf{s} dans $T_{segment}$}}
				\STATE $Ajout(\red{A},s)$
				\ENDFOR				
				\FORALL{\textit{segments \textbf{s} dans $Right$}}
					\STATE $Ajout(\red{A},s)$
				\ENDFOR
			\ENDIF
		\ENDIF
	\ENDIF
	\STATE \textit{\textbf{retourner} \red{A}}
\ENDIF
\end{algorithmic}
\end{algorithm}

\end{document}