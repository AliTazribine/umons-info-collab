\section{Transformée de Fourier Discrète}
    La TFD est définie par :
    $$
        X(k) = \sum_{n=0}^{N-1} x(n)e^{-jnk\frac{2\pi}{N}}
    $$

    On peut facilement la représenter sur un cercle de rayon unitaire.

    La TFD inverse est définie par :
    $$
        x(n) = \frac{1}{N}\sum_{k=0}^{N-1} X(k)e^{jnk\frac{2\pi}{N}}
    $$

    Calculer la TFD implique toujours de d'abord appliquer une fenêtre. Par défaut, il s'agit d'une fenêtre rectangulaire. Par conséquent, les valeurs de la TFD ne correspondent pas exactement à celles de la transformée de Fourier.

    \subsection{Fenêtres usuelles}\label{subsec:fenetres}
        Donnons d'abord les définitions des fenêtres :
        \begin{align*}
            \text{Rectangulaire} && w(n) &= \begin{cases}
                1 & \text{si } n \in [[0, N-1]] \\
                0 & \text{sinon}
            \end{cases}\\
            \text{Hanning} && w(n) &= \begin{cases}
                \frac{1}{2} - \frac{1}{2}\cos(\frac{2\pi n}{N}) & \text{si } n \in [[0, N-1]]\\
                0 & \text{sinon}
            \end{cases}\\
            \text{Hamming} && w(n) &= \begin{cases}
                0.54-0.46\cos(\frac{2\pi n}{N}) & \text{si } n \in [[0, N-1]]\\
                0 & \text{sinon}
            \end{cases}\\
            \text{Blackman} && w(n) &= \begin{cases}
                0.42-0.5\cos(\frac{2\pi n}{N})+0.08\cos(\frac{4\pi n}{N}) & \text{si } n \in [[0, N-1]]\\
                0 & \text{sinon}
            \end{cases}\\
        \end{align*}

        La \autoref{tab:fenetres} donne les caractéristiques des différentes fenêtres.

        \begin{table}
            \centering
            \begin{tabular}{lrr}
                \toprule
                Fenêtre & \makecell{Demi-largeur du lobe principal\\(en fréquence)} & \makecell{Niveau des lobes secondaires\\(en décibel)}\\
                \midrule
                Rectangulaire & $\frac{1}{N}$ & $-13\,dB$\\
                Hanning & $\frac{3}{2N}$ & $-30\,dB$\\
                Hamming & $\frac{2}{N}$ & $-40\,dB$\\
                Blackman & $\frac{11}{4N}$ & $-60\,dB$\\
                \bottomrule
            \end{tabular}
            \caption{Comparaison des fenêtres usuelles}
            \label{tab:fenetres}
        \end{table}