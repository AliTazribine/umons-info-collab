\chapter{Echantillonnage}
    Regardons maintenant comment passer d'un signal continu à un signal discret. Cette conversion est appelée \textit{échantillonnage}\index{Échantillonnage}.

    L'échantillonnage consiste à construire, à partir d'un signal analogique $f(t)$, un signal à temps discret $f(n) = f(nT_e)$ obtenu en mesurant la valeur de $f(t)$ toutes les $T_e$ secondes. On peut voire ceci comme le fait de multiplier $f(t)$ par un train d'impulsions de Dirac de période $T_e$ :
    $$
        f^+(t) = f(t)\delta_{T_e}(t) = \sum_n f(nT_e)\delta(t - nT_e)
    $$

    On peut alors interpréter la transformée de Fourier $F^+(f)$ de $f^+(t)$ (c'est-à-dire, la TFTD de $\{f(n)\}$) comme celle d'un produit (pour rappel, un produit en temps est une convolution en fréquence) :
    $$
        F^+(f) = F(f) * \left[\frac{1}{T}\delta_{f_e}(f)\right] = \frac{1}{T_e} \sumInfty{k} F(f - nf_e)
    $$

    Le schéma de la page 103 du syllabus illustre très bien ce principe.

    \begin{remarque}
        La transformée de Fourier obtenue est continue et périodique (ce qui est logique, vu que c'est un TFTD d'un signal non périodique).
    \end{remarque}

    \section{Recouvrement spectral (aliasing)}
        Si le spectre $F(f)$ du signal analogique $f(t)$ n'est pas nul au delà de $\frac{f_e}{2}$, la superposition peut conduire à des empiètements des translatées. Ce phénomène est appelé \textit{recouvrement} (ou \textit{repliement}) \textit{spectral} (ou, en anglais, \textit{aliasing})\index{Recouvrement spectral}\index{Repliement spectral|see {Recouvrement spectral}}\index{Aliasing|see {Recouvrement spectral}}.

        Ce recouvrement spectral a pour conséquence que le signal $f(n)$ n'est plus une image correcte de $f(t)$. On appelle ceci l'\textit{effet stroboscopique}\index{Effet stroboscopique}.

        Le terme de \textit{repliement spectral} vient du fait que tout se passe comme si la partie de $F(f)$ inférieure à $\frac{f_e}{2}$ se trouvait additionnée à la partie de ce même $F(f)$ supérieure à $\frac{f_e}{2}$, repliée autour de $\frac{f_e}{2}$ et conjuguée. La figure de la page 104 illustre ceci.

        On peut simplement corriger ce problème en prenant une fréquence d'échantillonnage au moins deux fois plus grande que la plus haute composante fréquentielle du signal.
        
        Cependant, certains signaux ont un spectre infini. Dans ce cas, on choisit la fréquence d'échantillonnage de façon à ce que le recouvrement spectral ne dépasse pas un certain seuil. On peut aussi ajouter un \textit{filtre de garde}\index{Filtre de garde} qui ne laissent passer que les fréquences inférieures ou égales à $\frac{f_e}{2}$. Evidemment, en faisant ceci, on commet toujours une erreur dans l'échantillonnage vu qu'on ne considère pas toutes les fréquences.

    \section{Théorème de Shannon}
        Le \textit{théorème de Shannon} (ou \textit{théorème de l'échantillonnage})\index{Théorème de Shannon}\index{Théorème de l'échantillonnage|see {Théorème de Shannon}} dit :
        \begin{theorem}[Théorème de Shannon]
            Tout fonction $f(t)$ dont le spectre est à support bornée (c'est-à-dire que $\exists f_M$ tel que $\forall |f| > f_M, F(f) = 0$) est complètement définie par ses échantillons $f(n T_e)$ si $f_e \geq 2f_M$.
        \end{theorem}

        Si on échantillonne en respectant ce théorème, il n'y a pas de recouvrement dans les échantillons.

        On peut reconstruire le signal d'origine en faisant une convolution dans le temps entre $f^*(t)$ et $sinc(\frac{t}{T_e})$ (une fonction pieuvre qui est considérée comme la \textit{fonction d'interpolation idéale}\index{Fonction d'interpolation idéale}). En fréquence, ceci nous donne un produit entre $F$ et $T_e rect(\frac{f}{T_e})$\footnote{On peut calculer cette transformée de Fourier en utilisant le fait que $sinc(t) \fourier rect(f)$ et la propriété de changement d'échelle qui est $f(at) \fourier \frac{1}{|a|}F(\frac{f}{a})$.}. Ceci est illustré (avec une erreur dans la fonction $sinc$) à la page 109.

    \section{Reconstruction du signal à temps continu}
        A partir des échantillons, on veut reconstruire le signal d'origine $f(t)$. Il faut procéder en deux étapes :
        \begin{enumerate}
            \item On construit un vrai signal analogique $f^*(t)$ à partir des échantillons du signal à temps discret. On appelle ceci \textit{extrapolation}\index{Extrapolation}. On peut voir $f^*(t)$ comme une première ébauche de $f(t)$.
            \item On applique sur $f^*(t)$ un \textit{filtre de lissage}\index{Filtre de lissage} qui affine $f^*(t)$ et le rapproche de $f(t)$.
        \end{enumerate}

        % TODO: si j'en vois dans les examens

    \section{Changement de fréquence d'échantillonnage}
        Il peut arriver qu'on souhaiter augmenter ou diminiuer la fréquence d'échantillonnage d'un signal déjà échantillonné. Ici, on va regarder comment multiplier ou diviser la fréquence d'échantillonnage par un nombre entier.

        \subsection{Décimation}
            Considérons un signal $x_1(n)$ obtenu par échantillonnage d'un signal analogique $x(t)$ à une fréquence d'échantillonnage $f_e$. Le spectre utile du signal est limité à l'intervalle $[0, \frac{f_e}{2}]$. On veut \textit{sous-échantillonner}\index{Sous-échantillonnage} ce signal à une fréquence $f_e'$ qui est $k$ fois inférieure à $f_e$. On parle aussi de \textit{downsampling}\index{Downsampling|see {Sous-échantillonnage}}.

            On va, dans un premier temps, prendre un échantillon sur $k$ de $x_1(n)$. On va \textit{décimer}\index{Décimation} $x_1(n)$ par $k$.

            Au départ, on avait un signal $x(t)$ échantillonné à $f_e$. On a ensuite appliqué un autre échantillonnage à $\frac{f_e}{k}$. Ceci est évidemment équivalent à un échantillonnage direct à $\frac{f_e}{k}$. Voici quelques résultats :
            \begin{itemize}
                \item Le spectre du signal décimé est celui du signal analogique de départ, rendu périodique de période $\frac{f_e}{k}$ (car TFTD).
                \item L'amplitude spectrale est divisée par $kT_e$ (car on divise par $k$ par rapport au spectre du signal échantillonné à $f_e$).
            \end{itemize}

            Cependant, une simple décimation ne suffit pas. En effet, comme il y a généralement un filtre de garde reglé pour un échantillonnage à une fréquence de $f_e$, le sous-échantillonnage par $k$ introduit un repliement spectral des composantes de $x_1(n)$ situées entre $\frac{f_e}{2k}$ et $\frac{f_e}{2}$. Il est donc nécessaire de faire précéder le sous-échantillonnage d'un filtre \textit{numérique} passe-bas de fréquence de coupure égale à $\frac{f_e}{2k}$. La décimation totale est illustrée à la page 113.

        \subsection{Interpoler}
            Considérons un signal $x_1(n)$ obtenu par échantillonnage d'un signal analogique $x(t)$ à une fréquence d'échantillonnage $f_e$. Le spectre utile du signal est limité à l'intervalle $[0, \frac{f_e}{2}]$. On veut \textit{sur-échantillonner}\index{Sur-échantillonnage} ce signal à une fréquence $f_e'$ qui est $k$ fois supérieure à $f_e$. On parle aussi de \textit{upsampling}\index{Upsampling|see {Sur-échantillonnage}}.

            Il faut calculer $k - 1$ échantillons intermédiaires entre deux échantillons de $x_1(n)$. Ceci est possible car, par le théorème de Shannon, on sait qu'on peut totalement reconstituer le signal analogique en utilisant l'interpolateur idéal.

            Pour faire ceci, il suffit simplement de rajouter les échantillons suplémentaires (et de mettre leur valeur à zéro) de la façon suivante :
            $$
                x_2(n) = \begin{cases}
                    x_1(\frac{n}{k}) &\text{si } n \text{ est mutliple de } k\\
                    0 &\text{sinon}
                \end{cases}
            $$

            Comme on rajouté uniquement des zéros, ceci ne modifie pas la TFTD, qui reste périodique de période $f_e$. Par contre, le spectre utile de $x_2(n)$ s'étend maintenant de 0 à $\frac{kf_e}{2}$. Ceci fait donc apparaître de nouvelles composantes de $x_1(n)$ (qu'on ne veut pas). On peut les éliminer en faisant suivre le sur-échantillonnage par un filtre passe-bas \textit{numérique} de fréquence de coupure égale à $\frac{f_e'}{2k}$. Le résultat est appelé \textit{interpolation}\index{Interpolation} du signal et est illustré à la page 114.

    \section{Filtre de garde réel et sur-échantillonnage (oversampling)}
        Le filtre de garde idéal est impossible à réaliser en pratique. On prendra donc un filtre de garde qui modifie légérement le spectre original autour de $f_M$ (à cause du repliement de composantes résiduelles supérieures à $\frac{f_e}{2}$).

        On peut compenser cet effet de deux façons :
        \begin{itemize}
            \item Si la fréquence d'échantillonnage est imposée, on peut faire en sorte que la bande passante du filtre soit plus étroite que la limite théorique de $\frac{f_e}{2}$. Ceci attenue les recouvrements spectraux MAIS les composantes à plus haute fréquence sont perdues !
            \item Si on peut choisir $f_e$, alors on peut la prendre fortement supérieure à $2f_M$. Ceci fait qu'on a plus de calculs à exécuter !
        \end{itemize}

    \section{Filtre de lissage réel et sur-échantillonnage (oversampling)}
        On peut simplifier le filtre de lissage en utilisant une fréquence d'échantillonnage intermédiaire, supérieure à la fréquence de départ, avant d'extrapoler les échantillons. Il suffit de faire du upsampling. Ceci fait qu'on a plus de calculs à exécuter au niveau du filtre numérique mais on gagne en simplification au niveau du filtre de lissage.

        \begin{exemple}
            L'exemple des CDs de la page 117 semble être assez important !
        \end{exemple}

    \section{Théorème de Shannon généralisé}
        On peut généraliser le théorème de Shannon\index{Théorème de Shannon} pour les signaux à bande étroite, c'est-à-dire pour les signaux pour lesquels l'amplitude spectrale se trouve confinée dans une bande de fréquence de largeur $B$ centrée autour de $f_0$. Ceci est illustré (et expliqué en détail) à la page 118. On ne donne ici que le théorème.

        \begin{theorem}[Théorème de Shannon généralisé]
            Toute fonction $f(t)$ dont le spectre est à bande étroite $(f_0, B)$ est complètement définie par ses échantillons $f(nT_e)$ si $f_e\geq 2B$ et que $f_e$ respecte en outre l'une des conditions suivantes : $Kf_e = f_0 - \frac{B}{2}$ ou $Kf_e = f_0 + \frac{B}{2}$, avec $K$ entier.
        \end{theorem}

        \begin{exemple}
            L'exemple des pages 119 à 121 semble également important !
        \end{exemple}