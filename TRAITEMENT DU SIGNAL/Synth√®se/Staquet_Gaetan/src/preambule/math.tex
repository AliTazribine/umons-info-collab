\usepackage{amsmath}
\usepackage{amsthm}
\usepackage{amssymb}
\usepackage{mathtools}

\newcommand{\N}{\mathbb{N}\xspace}

\renewcommand{\epsilon}{\varepsilon}
\renewcommand{\phi}{\varphi}

\allowdisplaybreaks % Autorise les environnements à faire un page break

% J'ai suivi les recommandations de http://mirror.koddos.net/CTAN/macros/latex/required/amscls/doc/amsthdoc.pdf
\theoremstyle{plain}
\newtheorem{thm}{Théorème}[section]
\newtheorem{theorem}[thm]{Théorème} % Parce qu'à chaque fois, je me trompe
\newtheorem{lemme}{Lemme}[section]
\newtheorem{corollaire}{Corollaire}[section]

\theoremstyle{definition}
\newtheorem{definition}{Définition}[section]
\newtheorem{propriete}{Propriété}[section]
\newtheorem{exemple}{Exemple}[section]

\theoremstyle{remark}
\newtheorem{note}{Note}[section]
\newtheorem{rem}{Remarque}[section]
\newtheorem{remarque}[rem]{Remarque}
\newtheorem{illustration}{Illustration}[section]

\newenvironment{demonstration}{\begin{proof}[\textnormal{\textbf{Preuve}}]}{\end{proof}}
\newenvironment{preuve}{\begin{demonstration}}{\end{demonstration}\ignorespacesafterend}

% Traductions
\newcommand{\thmautorefname}{Théorème}
\renewcommand{\theoremautorefname}{Théorème}
\newcommand{\proprieteautorefname}{Propriété}
\newcommand{\exempleautorefname}{Exemple}
\newcommand{\lemmeautorefname}{Lemme}
\newcommand{\definitionautorefname}{Définition}
\newcommand{\algorithmautorefname}{Algorithme}