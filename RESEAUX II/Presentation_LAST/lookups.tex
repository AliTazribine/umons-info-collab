\documentclass{beamer}
\usepackage[utf8]{inputenc}
\usepackage{color}
\usepackage{graphicx}
\usepackage[french]{babel}
\selectlanguage{french}
\definecolor{darkred}{rgb}{0.85,0,0}
\definecolor{darkblue}{rgb}{0,0,0.7}
\definecolor{darkgreen}{rgb}{0,0.6,0}
\definecolor{darko}{rgb}{0.93,0.43,0}
\definecolor{maintitle}{rgb}{0.66,0,0.22}
\definecolor{title}{rgb}{0,0.5,0.5}
\definecolor{greyy}{rgb}{0.6,0.6,0.6}

\newcommand{\image}[1]{\includegraphics{#1}}
\newcommand{\imageR}[2]{\includegraphics[width=#2px]{#1}}
\newcommand{\imageRT}[2]{\includegraphics[height=#2px]{#1}}
\newcommand{\img}[1]{\begin{center}\includegraphics[width=400px]{#1}\end{center}}
\newcommand{\imag}[1]{\begin{center}\includegraphics{#1}\end{center}}
\newcommand{\imgR}[2]{\begin{center}\includegraphics[width=#2px]{#1}\end{center}}
\newcommand{\imgRT}[2]{\begin{center}\includegraphics[height=#2px]{#1}\end{center}}
\newcommand{\term}[1]{\textit{\textcolor{maintitle}{#1}}}
\newcommand{\cerm}[1]{\textcolor{maintitle}{#1}}
\newcommand{\gre}[1]{\textcolor{darkgreen}{#1}}
\newcommand{\blu}[1]{\textcolor{darkblue}{#1}}
\newcommand{\red}[1]{\textcolor{darkred}{#1}}

\usetheme{Warsaw}
\usecolortheme{beaver}
%\setbeamercolor{section in head/foot}{bg=black}
\setbeamercolor{title}{bg=maintitle,fg=white}

\title[IP Address Lookup]{IP Address Lookup \\ $\sim$ \\ Structures et algorithmes}
\author{Destercq Lionel \& Dubuc Xavier}

\institute{\imageR{UMONS.jpg}{100}}

\addtobeamertemplate{footline}{\insertframenumber/\inserttotalframenumber}

\begin{document}

\AtBeginSection[]
{
  \begin{frame}<beamer>
    \frametitle{Plan}
    \tableofcontents[currentsection,currentsubsection]
  \end{frame}
}

\AtBeginSubsection[]
{
  \begin{frame}<beamer>
    \frametitle{Plan}
    \tableofcontents[currentsection,currentsubsection]
  \end{frame}
}

\begin{frame}
\titlepage
\end{frame}

%%%%%%%%%%%%%%%%%%%%%%%%%%%%%%%%%%%%%%%%%%%%%%%% INTRODUCTION %%%%%%%%%%%%%%%%%%%%%%%%%%%%%%%%%%%%%%%%%

\section{Introduction}

\begin{frame}{Introduction}

Pour commencer, quelques définitions afin de bien situer notre sujet :

\begin{block}{Lookup}
Le \term{lookup} d'une adresse IP est l'opération qu'effectue un routeur lorsqu'il consulte sa \textbf{table de forwarding} à la
recherche de l'entrée correspondant à cette adresse IP afin d'obtenir l'interface de sortie surlaquelle il doit placer le paquet
qu'il désire envoyer ainsi que le next-hop. Pour effectuer ce look-up, le routeur a recours à la méthode du \textbf{longest match 
prefix} qui sélectionne l'entrée de la table de forwarding qui a le plus de bits en commun avec le préfixe recherché.
\end{block}

\end{frame}

\begin{frame}{Exemple}
\begin{figure}
	\begin{center}
	\imageR{CN_001.png}{200}
	\caption{Exemple de table de forwarding.}
	\end{center}	
\end{figure}
\end{frame}

\begin{frame}{CIDR et agrégation}

\begin{block}{CIDR}
Le plan d'adressage "\textit{\term{C}lassless \term{I}nter-\term{D}omain \term{R}outing}" permet plus de flexibilité dans 
l'adressage des adresses IPv4 mais il occasionne une expansion de la taille des tables de forwarding \textit{(même si l'on peut 
contenir en partie cette expansion avec l'\textbf{agrégation de préfixes IP} mais pas dans tous les cas)}.
\end{block}

\begin{exampleblock}{Exemple : Agrégation impossible}
Imaginons qu'un client détenant le préfixe 208.12.21/24 change de provider sans changer de préfixe. Après ce changement, on ne 
peut plus agréger les préfixes allant de 208.12.16/24 à 208.12.31/24 car 208.12.21/24 n'est plus joignable via le même provider
que les autres préfixes.
\end{exampleblock}

\end{frame}

\begin{frame}{Implémentation}

Les \textbf{tables de forwarding} peuvent être implémentées de différentes manières, chacune de ces manières possèdant ses 
avantages et ses inconvénients. Nous allons voir quelques-unes de ces méthodes et nous allons les comparer selon les critères 
suivants : \\
\begin{itemize}
\pause \item \cerm{Vitesse de lookup} : \textit{au moins assez pour suivre la fréquence d'entrée des paquets}.
\pause \item \cerm{Exigence de stockage} : \textit{moins de données stockées implique :}
	\begin{itemize}
	\item \textbf{accès mémoire rapides,}
	\item \textbf{diminution de l'utilisation d'énergie}.
	\end{itemize}
\pause \item \cerm{Temps d'update} : \textit{les routeurs doivent être capables de supporter 1000 BGP-Update par seconde}.
\pause \item \cerm{Extensibilité} : \textit{la taille de la table de forwarding peut augmenter de 25000 entrées par an}.
\end{itemize}

\end{frame}

\begin{frame}{Implémentation}

Nous avons classés ces algorithmes en 3 catégories :
\begin{enumerate}
\pause \item les algorithmes basés sur les arbres de préfixes,
\item les algorithmes de recherches binaires (ou par bissection),
\item les algorithmes/schémas basés sur le hardware.
\end{enumerate}

\end{frame}

%%%%%%%%%%%%%%%%%%%%%%%%%%%%%%%%%%%%%%%%%%%% TRIE BASED ALGORITHMS %%%%%%%%%%%%%%%%%%%%%%%%%%%%%%%%

\section{Algorithmes basés sur les arbres de préfixes}

%%%%%%%%%%%%%%%%%%%%%%%%%%%%%%%%%%%%%%%%%% BINARY TRIE %%%%%%%%%%%%%%%%%%%%%%%%%%%%%%%%%%%%%%%

\subsection{Arbres binaires (Binary Trie)}

\begin{frame}{Définition de la structure}

Il s'agit de la méthode la plus intuitive, on stocke les préfixes de la base de données dans un arbre binaire dont le parcours de 
chaque branche correspond à lire un 1 ou un 0. On choisit l'une de ces 2 branches en fonction du préfixe que l'on cherche à
résoudre.

\begin{figure}
	\begin{center}
	\imageR{CN_002.png}{100}
	\caption{Exemple d'arbre binaire.}
	\end{center}	
\end{figure}

\end{frame}

\begin{frame}{Exemple de lookup}

\begin{figure}
	\begin{center}
	\imageR{CN_003.png}{100}
	\caption{Exemple d'arbre binaire.}
	\end{center}	
\end{figure}

Si l'adresse de destination donnée est \textbf{11101000}, le \cerm{lookup} commence à la racine, puis se poursuit à 
droite, puis encore à droite, ... jusqu'à atteindre le noeud contenant le préfixe P7 qui est retourné comme résultat du lookup.\\
\textit{(Ce noeud contient les informations de next-hop)}
\end{frame}

\begin{frame}{Exemple de lookup 2}

\begin{figure}
	\begin{center}
	\imageR{CN_002.png}{100}
	\caption{Exemple d'arbre binaire.}
	\end{center}	
\end{figure}

Si l'adresse de destination donnée est \textbf{111000}, que se passe-t-il ?

\end{frame}

\begin{frame}{Exemple de lookup 2}

\begin{figure}
	\begin{center}
	\imageR{CN_004.png}{100}
	\caption{Exemple d'arbre binaire.}
	\end{center}	
\end{figure}

L'algorithme doit tenir en mémoire le dernier préfixe visité dans le cas où son chemin ne l'emmène pas vers un noeud contenant un 
préfixe.

\end{frame}

\begin{frame}{Complexités}
Soit \gre{$W$} la taille maximale d'un préfixe et \red{$N$} le nombre de préfixes contenu dans la table de forwarding, \\
\imageR{good.png}{15}
\begin{itemize}
\item Intuitive, simple.
\item Les complexités d'un \cerm{lookup} et d'un \cerm{update} en temps dans le pire des cas sont en \gre{$O(W)$}. \\
\end{itemize}
\imageR{bad.png}{15}
\begin{itemize}
\item Dans le pire des cas, il faut \red{$32$ accès mémoire} par lookup.
\item Dans le pire des cas, l'ajout d'un préfixe nécessite l'ajout de \red{$32$ noeuds}.
\item La mémoire nécessaire est en \red{$O(N\times W)$}.
\end{itemize}
\end{frame}

\begin{frame}{Variantes}
\begin{enumerate}
\item L'arbre peut être étendu à un arbre complet de telle sorte à ce que les préfixes soient tous stockés dans les feuilles, la 
structure devient en fait une structure plate. \\ \red{Si les adresses font 32bits, $2^{32}$ entrées sont 
nécéssaires ! \\ $\Rightarrow$ Impraticable.}
\item Le "\cerm{\textbf{leaf pushing}}".
\end{enumerate}
\end{frame}

\begin{frame}{Idées clés du Leaf pushing}
\begin{enumerate}
\item Transformer un ensemble donné de préfixes en un ensemble de préfixes disjoints.
\item L'arbre correspondant comportera ses préfixes dans les feuilles et aucun dans les noeuds internes \textit{(d'où le nom de la 
technique, les préfixes étant "poussés" vers les feuilles)}.
\item Pour créer l'arbre, on ajoute une feuille à chaque noeud ne possédant qu'un seul fils et on y stocke le préfixe de l'ancètre 
le plus proche. On supprime également tous les préfixes stockés dans les noeuds internes.
\end{enumerate}
\end{frame}

\begin{frame}{Exemple de Leaf pushing}

\begin{figure}
	\begin{center}
	\imageR{CN_005.png}{225}
	\caption{Exemple d'arbre "leaf pushed".}
	\end{center}	
\end{figure}

\end{frame}

\begin{frame}{Différences avec les arbres binaires}
Les seules différences notables sont :
\begin{enumerate}
\pause \item \gre{on n'utilise pas la règle \textit{"longest match prefix"} mais on trouve tout de même le préfixe le 
plus spécifique,}
\item \red{certains préfixes doivent être stockés plusieurs fois \\ $\Rightarrow$ augmentation de la mémoire nécessaire.}
\end{enumerate}
\end{frame}

%%%%%%%%%%%%%%%%%%%%%%%%%%%%%%%%%%% PATH-COMPRESSED %%%%%%%%%%%%%%%%%%%%%%%%%%%%%%%%%%%%%%%%%

\subsection{Arbres aux chemins compressés (Path-compressed)}

\begin{frame}{Introduction de la structure}

L'idée de ces arbres est de réduire la hauteur en enlevant quelques noeuds internes et en stockant 2 infos supplémentaires par 
noeud restant : la \red{skip value} et \blu{segment}.

\begin{block}{Skip value}
Nombre indiquant le nombre de bits qui ont été passés sur le chemin.
\end{block}

\begin{block}{Segment}
Chaîne de caractères de longueur variable indiquant la chaîne de digits passés sur le chemin.
\end{block}

\end{frame}

\begin{frame}{Exemple}

\begin{figure}
	\begin{center}
	\imageR{CN_006.png}{225}
	\caption{Exemple d'arbre "path compressed".}
	\end{center}	
\end{figure}

En \textbf{P9} la \red{skip value} est de 2 car on a passé les deux bits "1" sur le chemin vers ce noeud (\blu{segment} est donc 
égal à "11", il faudra matcher cette chaîne pour atteindre \textbf{P9}).

\end{frame}

\begin{frame}{Complexités}

\imageR{good.png}{15} \\
Le stockage ne dépend plus de la longueur maximale de préfixe mais uniquement du nombre de noeuds de l'arbre, dénoté \textbf{N}.
$\Rightarrow \gre{O(N)}$\\$ $\\
\imageR{bad.png}{15}\\
Dans le cas d'un arbre plein, l'arbre binaire n'est pas modifié.\\ $\Rightarrow$ \red{mêmes complexités}
\end{frame}

%%%%%%%%%%%%%%%%%%%%%%%%%%%%%%%% BINARY SEARCH SUR LA LG DES PREFIXES %%%%%%%%%%%%%%%%%%%%%%%%%%%%%%%%

\section{Recherche binaire sur la longueur des préfixes}

\begin{frame}{Introduction de la structure}

On peut diviser l'opération du \textbf{longest match prefix} en une succession de \term{matching} \textbf{exacts}. \\
La structure de données utilisée est un ensemble de sous-tables $H_1$, $H_2$, ..., $H_W$ de telle sorte que $H_i$ contienne tous
les préfixes de taille $i$. Le matching va consister alors en une recherche dichotomique sur ces tables (et donc sur l'espace des 
longueurs de préfixe).

\end{frame}

\begin{frame}{Matching exact}
Afin de réduire le temps pris par le \term{matching exact}, on utilise une \textbf{fonction de hashage} \textit{(chaque sous-table 
utilise sa propre fonction)}. \\

Pour simplifier, nous considérerons qu'il n'y a pas de \textbf{hash collision} \textit{(2 préfixes hashés sur la même valeur)}.
\end{frame}

\begin{frame}{Lookup}
Le \term{lookup} commence par considérer d'abord la table $H_{\frac{W}{2}}$, de là 2 cas possibles :
\begin{itemize}
\item \gre{un noeud correspondant au préfixe est trouvé}, dans ce cas on peut ignorer toutes les sous-tables $H_i$ avec $i < 
\frac{W}{2}$ \textit{(longest matching prefix)}, on continue donc la recherche dans les sous-tables $H_j$ avec $j > \frac{W}{2}$.
\item \red{aucun noeud n'est trouvé}, dans ce cas on peut ignorer toutes les sous-tables $H_i$ avec $i > \frac{W}{2}$, on continue 
donc la recherche dans les sous-tables $H_j$ avec $j < \frac{W}{2}$.
\end{itemize}
etc...
\end{frame}

\begin{frame}{Exemple}
\begin{figure}
	\begin{center}
	\imageR{CN_008.png}{50}$\qquad$
	\imageR{CN_007.png}{200}
	\caption{Exemple de structure.}
	\end{center}	
\end{figure}

$H_{16}$ contient un \term{marqueur}, c'est-à-dire un préfixe n'existant pas dans la \textbf{FIB}, il est utilisé pour aider à 
déterminer la branche à choisir lors de la recherche.

\end{frame}

\begin{frame}{Complexités}
Soit \gre{W} la taille maximale d'un préfixe, \\$ $ \\
\imageR{good.png}{15}
\begin{itemize}
\item Update en \gre{$O(\log_2 W)$}
\item Lookup en \gre{$O(\log_2 W)$}
\end{itemize}
\imageR{bad.png}{15}
\begin{itemize}
\item Mémoire nécessaire en \red{$O(N \log_2 W)$}\\ \textit{(naïvement c'est en $O(NW)$ car il peut y avoir jusqu'à $W$ marqueurs 
par préfixe de la \textbf{FIB} mais au final seul $\log_2 W$ d'entre eux sont nécessaires (ceux qui seront testés par 
l'algorithme))}
\end{itemize}
\end{frame}

%%%%%%%%%%%%%%%%%%%%%%%%%%%%%%%%%%%%%%%%%%%%%%%% HARDWARE %%%%%%%%%%%%%%%%%%%%%%%%%%%%%%%%%%%%%%%%%%%%%%

\section{Schéma basé sur le hardware: TCAM}
\subsection{Ternary CAM} 
\begin{frame}{Ternary CAM}

\begin{itemize}
\item \textbf{T}ernary \textbf{C}ontent-\textbf{A}dressable \textbf{M}emory
\item C'est un type de mémoire utilisée pour des recherches à haute vitesse. \\
Plus connu en tant que mémoire associative.
\pause \item Dispose de 3 états de matching: 
\begin{itemize}
\item OK
\item KO
\item X (valeur "don't care")
\end{itemize}
\end{itemize}
\pause
\begin{exampleblock}{Exemple 1}
10XX1 'match' les mots 10001, 10011, 10101, 10111. 
\end{exampleblock}
\end{frame}
\begin{frame}{TCAM: Fonctionnement}
TCAM sauve chaque entrée comme un couple (valeur, mask). \\
La valeur mask est utile car elle sert à savoir si un bit 
est à considérer ou non dans la comparaison. \\
\pause
\begin{exampleblock}{Exemple 2}
Pour un mot de taille = 6, le préfixe 110* est noté comme étant le couple 
(110000,111000)
\end{exampleblock}
\end{frame}

\begin{frame}{TCAM: dans un routeur}
\begin{figure}
	\begin{center}
	\imageR{TCAM.png}{250}
	\caption{Schéma d'une table de forwarding utilisant TCAM.}
	\end{center}	
\end{figure}
\end{frame}

\begin{frame}{TCAM: dans un routeur}
Le fonctionnement est semblable : 
\begin{itemize}
\item Comparaison de l'adresse IP destination avec toutes les entrées (bit à bit et simultanément)
\pause \item Plusieurs matchs possibles:
\begin{itemize}
\item Notion de priorité selon la localisation dans la mémoire
\item Entrées classées selon les longueurs de préfixes
\end{itemize}
\end{itemize}

Ainsi le match dont le préfixe est le plus grand est selectionné. \\
\end{frame}

\begin{frame}{TCAM: dans un routeur}
\begin{itemize}
\item Espaces de la mémoire vide en cas d'ajout de préfixes.
\item Route par défaut: 
\begin{itemize}
\item Stockée au début de la mémoire
\item Valeur du mask = 0
\item Empruntée seulement s'il n'y a aucun autre match
\end{itemize}
\end{itemize}
\end{frame}
\begin{frame}{TCAM: performances}
\imageR{good.png}{15}
\begin{itemize}
\item TCAM n'a besoin que d'un accès mémoire pour son lookup 
\item Souvent utilisé car assez simple d'implémentation
\item $\sim$133 millions de lookup par seconde 
\end{itemize}
\pause
\imageR{bad.png}{15} \\
Néanmoins, TCAM souffre d'un haut ratio coût/densité et d'une consommation 
énergétique assez élevée (10-15W/puce). \\
$\rightarrow$ On va tenter de réduire la taille de la TCAM.
\end{frame}

\subsection{Réduction du coût mémoire}
\begin{frame}{Prefix Pruning}
Littéralement "élagage de préfixe"; on va tenter de supprimer les informations redondantes. \\
Mise en situation : 
\begin{figure}
	\begin{center}
	\imageR{prunning_ex.png}{150}
	\caption{Exemple de préfixes redondants.}
	\end{center}	
\end{figure}
\end{frame} 
\begin{frame}{Prefix Pruning}
Si on suppose que P1 et P2 ont la même information de forwarding (ie. P1 et P2 forwardent sur le même port), 
alors P2 est redondant. \\
$\rightarrow$ pour un longest match prefix qui devait se terminer en P2 elle se terminera en P1 si 
P2 est 'élagué'. \\
En réalité, les tables de forwarding ont un certain nombre de préfixes redondants. \\
$\rightarrow$ on élague et donc on réduit les entrées. \\

$\sim$20-30 \% de préfixes redondants
\end{frame}

\begin{frame}{Mask Extension}
On va utiliser la propriété principale des TCAM: 
\begin{itemize} 
\item Jusqu'à présent: 
\begin{itemize}
\item Uniquement utilisées pour des préfixes (ce qui implique?)
\item Voir le second exemple 
\end{itemize}
\pause \item Maintenant:
\begin{itemize}
\item On veut étendre le type 'prefix-match' à un type 'ternary-match'
\item Voir le premier exemple
\end{itemize}
\end{itemize}
\end{frame}

\begin{frame}{Mask Extension}
On va maintenant voir par un exemple comment passer d'un prefix-match à un 
ternary-match: 
\begin{figure}
	\begin{center}
	\imageR{mask.png}{200}
	\caption{Exemple de table de forwarding.}
	\end{center}	
\end{figure}
\end{frame}
\begin{frame}{Performances}
Grâce à ces deux techniques, typiquement, la taille des tables de forwarding est de 
$\sim$ 50-60\% par rapport à celle des tables sans optimisation.  \\
\pause Conséquences de ces deux techniques : 
\begin{itemize}
\item Prefix Pruning: aucun problème
\pause \item Mask Extension: \begin{itemize}
\item augmente la complexité des mises à jours
\item beaucoup de préfixes sont associés avec d'autres \pause $\rightarrow$ 
cela constitue un obstacle aux mises à jour incrémentales. 
\end{itemize}
\end{itemize}
\end{frame}

\begin{frame}{Pruning \& Mask Extension: Exemple}
\begin{figure}
	\begin{center}
	\imageR{norm_to_prun.png}{200}
	\caption{Exemple de réduction d'une table.}
	\end{center}	
\end{figure}

\end{frame}

\begin{frame}{Pruning \& Mask Extension: Exemple}
\begin{figure}
	\begin{center}
	\imageR{prun_to_mask.png}{200}
	\caption{Exemple de réduction d'une table.}
	\end{center}	
\end{figure}

\end{frame}
%%%%%%%%%%%%%%%%%%%%%%%%%%%%%%%%%%%%%%%%%%%%%%%% IPv6 %%%%%%%%%%%%%%%%%%%%%%%%%%%%%%%%%%%%%%%%%%%%%%

\section{Un mot sur les lookups IPv6}

\subsection{Caractéristiques}

\begin{frame}{IPv6 : Rappel}
\begin{figure}
	\begin{center}
	\imageR{CN_009.png}{200}
	\caption{Structure d'une adresse IPv6}
	\end{center}	
\end{figure}

\begin{itemize}
\item les adresses seront données par préfixe /48 (une très grosse organisation pourra éventuellement avoir des /47 ou plus 
petit encore voire plusieurs /48),
\item les /64 seront donnés lorsqu'il sera certain qu'un seul sous-réseau sera nécéssaire,
\item les /128 seront données lorsqu'il sera certain qu'un seul appareil sera nécéssaire.
\end{itemize}

\end{frame}

\begin{frame}{Caractéristiques intéressantes}

Des informations précédentes, on peut tirer les caractéristiques suivantes : 
\begin{enumerate}
\item \textbf{Aucun préfixe} de longueur comprise entre \textbf{64} et \textbf{128} bits ne sera utilisé.
\pause \item La plupart des préfixes seront des \textbf{/48} et, dans une moindre mesure des \textbf{/64}.
\pause \item Très peu de préfixes \textbf{/128} \textit{(tout comme les /32 pour \term{IPv4})}.
\end{enumerate}

\end{frame}

\subsection{Les TCAM pour IPv6}

\begin{frame}{IPv6 : TCAM}

Les TCAM supportent des mots de longueurs 36, 72, 144, 288 et 576 bits. \\
On est tenté d'utiliser des mots de longueurs = 144 bits. \\
Cependant, on a vu que dans IPv6 beaucoup de préfixes ont une longueur inférieure à 64 bits \\
\pause $\rightarrow$ plus de 50\% d'un mot stockera la valeur 'don't care'. 

On va présenter une approche afin de gagner 40\% d'espace. 

\end{frame}

\begin{frame}{IPv6 : TCAM}

Dans cette approche: 
\begin{itemize}
\item les préfixes sont divisés en deux groupes, $G_S$ et $G_L$.
\begin{itemize}
\item $G_S$ contient les préfixes avec moins de 72 bits. 
\item $G_L$ le reste
\end{itemize}
\pause \item Les blocs TCAM sont divisés en deux partitions, $P_S$ et $P_L$
\begin{itemize}
\item Les préfixes dans $G_S$ sont stockés dans $P_S$
\item Les préfixes dans $G_L$ sont stockés dans $P_S$ (les 72 premiers bits) et servent de marqueur
 et dans $P_L$, les bits restants concaténés à un tag de 16 bits. 
\end{itemize}
\pause \item Stockage sous forme d'un 6-uplet (préfixe, p, m, v, Next Hop, T) où p et m désignent si l'entrée
est un préfixe, un marqueur ou les deux, v si le nexthop est valide (v=1) ou pas (v=0) et T un tag qui désigne un 
sous groupe. 
\end{itemize} 

\end{frame}

\begin{frame}{IPv6 : TCAM - Algorithme}
\begin{enumerate}
\item Les 72 bits significatifs d'une adresse destination A sont extraites et on cherche avec ces 72 bits dans $P_S$.
\pause \item Si le meilleur match n'est pas un marqueur, alors on a le next hop.
\pause \item Sinon on extrait les 56 bits restant de A et on recherche ces 56 bits concaténés avec le tag
de 16 bits associé dans $P_L$. 
\pause \item Si le résultat de la recherche est invalide (v = 0), alors le paquet est forwardé au next hop 
trouvé à l'étape 2.
\pause \item Sinon, le paquet est forwardé en utilisant le next hop trouvé à l'étape 3. 
\end{enumerate}
\end{frame}

\begin{frame}{IPv6 : TCAM - Un exemple}
Soit la table suivante dont les adresses sont sur 12 bits, 8 bits pour $P_S$, 
et le tag est sur 4 bits. \\
Considérons que l'on veuille chercher l'adresse destination 1011.0110.1000 dans cette table.
\begin{figure}
	\begin{center}
	\imageR{CN_010.png}{200}
	\caption{Exemple de table de forwarding.}
	\end{center}	
\end{figure}

\end{frame}

\begin{frame}{IPv6 : TCAM - Un exemple}
Soit la table suivante dont les adresses sont sur 12 bits, 8 bits pour $P_S$, 
et le tag est sur 4 bits. \\
Considérons que l'on veuille chercher l'adresse destination 1011.0110.1000 dans cette table.
\begin{figure}
	\begin{center}
	\imageR{CN_011.png}{200}
	\caption{Exemple de table de forwarding.}
	\end{center}	
\end{figure}

\end{frame}
\section{Références}
\begin{frame}{Références (1/3)}
\begin{itemize}

\item V. Fuller, T. Li, J. Yu, and K. Varadhan, “Classless inter-domain routing (CIDR): an address
assignment and aggregation strategy,” RFC 1519 (Proposed Standard), Sept. 1993. [Online].
Available at: http://www.ietf.org/rfc/rfc1519.txt
\item D. Knuth, Fundamental Algorithms Vol. 3: Sorting and Searching. Addison-Wesley,
Massachusetts, 197
\item V. Srinivasan and G. Varghese, “Faster IP lookups using controlled prefix expansion,” in Proc.
ACM SIGMATICS, Madison, Wisconsin, pp. 1–10 (June 1998)
\item D. R. Morrison, “PATRICIA - Practical algorithm to retrieve information coded in alfanumeric,”
IEEE/ACM Transactions on Networking, vol. 17, no. 1, pp. 1093–1102 (Oct. 1968).

\end{itemize}
\end{frame}

\begin{frame}{Références (2/3)}
\begin{itemize}

\item M. Waldvogel, G. Varghese, J. Turner, and B. Plattner, “Scalable high-speed IP routing lookups,”
in Proc. ACM SIGCOMM, Cannes, France, pp. 25–36 (Sept. 1997).
\item CYNSE10512 Network Search Engine, CYPRESS, Nov. 2002. [Online]. Available at: http://
www.cypress.com
\item Ultra9M – Datasheet from SiberCore Technologies. [Online]. Available at: http://www.
sibercore.com
\item P. Gupta, “Algorithmic search solutions: features and benefits,” in Proc. NPC-West 2003,
San Jose, California (Oct. 2003).

\end{itemize}
\end{frame}

\begin{frame}{Références (3/3)}
\begin{itemize}

\item H. Liu, “Routing table compaction in ternary CAM,” IEEE Micro, vol. 22, no. 1, pp. 58–64
(Jan. 2002)
\item D. Pao, “TCAM organization for IPv6 address lookup,” in Proc. IEEE Int. Conf. on Advanced 
Communications Technology, Phoenix Park, South Korea, vol. 1, pp. 26–31 (Feb. 2005).

\end{itemize}
\end{frame}
\end{document}
