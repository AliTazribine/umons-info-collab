\documentclass[a4paper,11pt]{report}
 
\usepackage[utf8]{inputenc}  
\usepackage[T1]{fontenc}
\usepackage[francais]{babel}
\usepackage{amsfonts}
\usepackage{amsmath}
\usepackage{listings}
\usepackage{fullpage}
\usepackage[hidelinks]{hyperref}
\usepackage{fancyhdr}
\pagestyle{fancy}

\renewcommand{\thesection}{}
\renewcommand{\thesubsection}{}

\renewcommand{\headrulewidth}{0pt}
\fancyhead[C]{} 
\fancyhead[L]{}
\fancyhead[R]{}

\renewcommand{\footrulewidth}{1pt}
\fancyfoot[C]{\textbf{\thepage}} 
\fancyfoot[L]{Delplanque Julien}
\fancyfoot[R]{2013-2014}

\begin{document}
\renewcommand{\labelitemi}{$\cdot$}
\begin{Large}\begin{center} 
   \underline{\textbf{Trouver le $S$ permettant de résoudre $(\sqrt[n\,]{x}-\sqrt[n\,]{y})*S = x-y$}} 
\end{center}\end{Large}
\begin{itemize}
	\item On sait que $a^n-b^n = (a-b) \sum\limits_{k=0}^{n-1}{a^{n-1-k}b^k}$ (cf Internet)\\
	
	\item En remplaçant $a$ par $x^{1/n}$ et $b$ par $y^{1/n}$\\
	
	\item On a alors:\\
	\begin{center}
	${(x^{1/n})}^n-{(y^{1/n})}^n = (x^{1/n}-y^{1/n})\sum\limits_{k=0}^{n-1}{(x^{1/n})^{n-1-k}(y^{1/n})^k}$\\
	i.e $x-y = (\sqrt[n\,]{x}-\sqrt[n\,]{y})\sum\limits_{k=0}^{n-1}{\sqrt[n\,]{x^{n-1-k}}\sqrt[n\,]{y^k}}$\\
	\end{center}
	
	\item Ainsi, le $S$ recherché est $\sum\limits_{k=0}^{n-1}{\sqrt[n\,]{x^{n-1-k}}\sqrt[n\,]{y^k}}$
	
\section{Sources:}
\begin{itemize}
	\item \url{http://villemin.gerard.free.fr/Wwwgvmm/Identite/IdentAut.htm#idform}
	\item \url{http://villemin.gerard.free.fr/Wwwgvmm/Decompos/Divanmbn.htm}
\end{itemize}
\end{itemize}
\end{document}