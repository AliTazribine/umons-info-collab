\documentclass[a4paper,11pt]{report}
 
\usepackage[utf8]{inputenc}  
\usepackage[T1]{fontenc}
\usepackage[francais]{babel}
\usepackage{amsfonts}
\usepackage{amsmath}
\usepackage{listings}
\usepackage{fullpage}
\usepackage{fancyhdr}
\pagestyle{fancy}

\renewcommand{\thesection}{}
\renewcommand{\thesubsection}{}

\renewcommand{\headrulewidth}{0pt}
\fancyhead[C]{} 
\fancyhead[L]{}
\fancyhead[R]{}

\renewcommand{\footrulewidth}{1pt}
\fancyfoot[C]{\textbf{\thepage}} 
\fancyfoot[L]{Delplanque Julien}
\fancyfoot[R]{2013-2014 last update: 04/2015}

\begin{document}
\renewcommand{\labelitemi}{$\cdot$}
\begin{Large}\begin{center} 
   \underline{\textbf{Analyse I (Partie B): Les équations différentielles ordinaires}} 
\end{center}\end{Large}

\section{Définition}
\begin{itemize}
	\item Une EDO (d'ordre n) est une équation de type $f(t,\partial_t^nu,...,\partial_t^1u,u)=0$
	\item Une solution pour une EDO de ce type est une fonction $u:I\rightarrow\mathbb{R}$ où $I\subseteq\mathbb{R}$ qui satisfait l'EDO i.e $u$ est $n$ fois dérivable sur $I$ et $\forall t \in I,f(t,\partial_t^nu,...,\partial_t^1u,u)=0$
\end{itemize}

\section{Les EDO linéaires homogènes}
\begin{itemize}
	\item Forme générale: Une EDO linéraire homogène est de la forme:\\
	\begin{center}
	$\sum\limits_{i=0}^n a_i(t)\partial_t^iu(t)=0$
	\end{center}
	\item Exemple: $\sin(t)\partial_t^3u(t)+t^2\partial_t u(t)-5 u(t)=0$
	\item Rappel:
	\begin{itemize}
		\item $\partial^k(u_1(t)+u_2(t)) = \partial^k u_1(t) + \partial^k u_2(t)$
		\item $\partial^k(\alpha u_1(t)) = \alpha \partial^k u_1(t)$
		\item $\partial^k$ est un opérateur linéaire.
	\end{itemize}
\end{itemize}

\section{Conséquences}
Si 
\begin{itemize}
\item $u_1(t)$ est solution de $\sum\limits_{i=0}^n a_i(t) \partial_t^i u(t) = 0$ et 
\item $u_2(t)$ est solution de $\sum\limits_{i=0}^n a_i(t) \partial_t^i u(t) = 0$, 
\end{itemize}

alors 
\begin{itemize}
\item $u_1(t) + u_2(t)$ est solution de $\sum\limits_{i=0}^n a_i(t) \partial_t^i u(t) = 0$
\item $\alpha u_1(t)$ est solution de $\sum\limits_{i=0}^n a_i(t) \partial_t^i u(t) = 0$.
\end{itemize}

\section{Principe de superposition}
Si on connait une solution particulière de l'équation notée $u_p(t)$ et si $u_0(t)$ est solution de l'EDO homogène, alors les solutions de l'EDO seront de la forme $u(t) = u_p(t) + u_0(t)$.

\newpage
\section{Résoudre une EDO}
Par principe de superposition, on va devoir trouver une solution particulère et une solution de l'équation homogène.
\subsection{Résoudre une équation homogène}
\begin{itemize}
	\item Trouver le polynome caractéristique $\sum\limits_{i=0}^n a_i(t) x^i$.
	\item Trouver les racines du polynome caractéristique et leurs multiplicité.
	\item Les solutions du polynôme de la forme $\sum\limits_{i} P_i(t) e^{\lambda it}$ où $\lambda i$ est la racine du polynome caractéristique et $P_i(t)$ est le polynôme de degré $<m_i$ qui est la multiplicité de la racine $\lambda_i$.
\end{itemize}

\subsection{Trouver une solution particulière de l'équation $\sum\limits_{i=0}^n a_i(t)\partial_t^iu(t)=q(t)e^{ut}$ où $u\in\mathbb{R}$, $q(t)$ est un polynôme.}
Il y a deux cas:
\begin{itemize}
	\item Si $u$ est racine du polynome caractéristique, alors il existe une solution de la forme $t^m r(t) e^{ut}$ où m est la multiplicité de $u$ et $r(t)$ est un polynome de degré $\le$ au degré de $q(t)$.
	\item Si $u$ n'est pas une racine du polynome caractéristique, alors il existe une solution particulière de la forme $r(t) .e^{ut}$ où $r(t)$ est un polynome de degré $\le$ degré $q(t)$
\end{itemize}

\section{Remarque}
\begin{itemize}
\item Pour trouver les solution réelles de l'EDO, il faut prendre la partie réelle des solution $\mathbb{C}$.
\item $cos(t)=\frac{e^{it}+e^{-it}}{2}$
\item $sin(t)=\frac{e^{it}-e^{-it}}{2i}$
\item $cosh(t)=$
\item $sinh(t)=$
\end{itemize}
\end{document}