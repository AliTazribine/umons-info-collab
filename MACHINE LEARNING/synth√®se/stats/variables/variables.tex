\subsection{Variables aléatoires et distributions}
    \begin{definition}
        Une \textit{variable aléatoire} est une fonction \(\Omega \to \R\). 
        
        Une façon de voir une variable aléatoire est de penser à un mapping entre une distribution sur \(\Omega\) et une distribution sur les réels (c'est-à-dire l'ensemble des valeurs de la variable aléatoire).
        Formellement, pour une variable \(X\) et un sous-ensemble \(A \in \R\) :
        \[
            \distribution_X (X \in A) = \distribution(\{\omega \in \Omega : X(\omega) \in A\})
        \]
    \end{definition}

    \begin{definition}
        Chaque variable aléatoire est associée à une \textit{fonction de distribution cumulative} (notée CDF) :
        \[
            \forall x, F_X(x) = \distribution_X(X \leq x)
        \]

        Une fonction \(F\) est une CDF si et seulement si :
        \begin{enumerate}
            \item \(\lim_{x \to -\infty} F(x) = 0\) et \(\lim_{x \to +\infty} F(x) = 1\)
            \item La fonction n'est pas décroissante en \(x\)
            \item La CDF est continue à droite, c'est-à-dire, \(\forall x_0 \in \R, \lim_{x \to x_0^+} F(x) = F(x_0)\)
        \end{enumerate}

        \(X\) est une variable continue si sa CDF est une fonction continue et est une variables discrète si sa CDF est une fonction discrète.

        Deux variables aléatoires \(X\) et \(Y\) sont identiquement distribués si \(\forall A, \distribution_X(X \in A) = \distribution_Y(Y \in A)\) (ne veut pas dire que \(X\) et \(Y\) sont égaux).

        Deux variables \(X\) et \(Y\) sont identiquement distribués si et seulement si leur CDF sont égaux, c'est-à-dire, \(\forall x, F_x(x) = F_y(x)\)
    \end{definition}

    \begin{remarque}[Notations]
        \(F_x(x)\) indique une CDF tandis que \(f_X(x)\) indique une pdf/pmf.
    \end{remarque}

    \begin{definition}
        Pour une variable discrète, sa \textit{fonction de masse} (notée \textit{PMF}) :
        \[
            f_X(x) = P_X(X = x)
        \]

        Pour une variable continue, sa \textit{densité de probabilité} (notée \textit{PDF}) \(f_X\) est la fonction qui satisfait :
        \[
            \forall x, F_X(x) = \int_{-\infty}^{x} f_X(t)\,dt
        \]

        Une fonction \(f_X(x)\) est une pdf/pmf si et seulement si :
        \begin{enumerate}
            \item \(\forall x, f_X(x) \geq 0\)
            \item \(\sum_x f_X(x) = 1\) (pour une pmf) ou \(\int_{-\infty}^{+\infty} f_X(x)\,dx = 1\) (pour une pdf)
        \end{enumerate}
    \end{definition}

    Pour trouver la probabilité qu'une variable aléatoire atterrisse dans un intervalle, il y a deux façons :
    \begin{itemize}
        \item Via les fonctions de distribution : \(\distribution(a < X \leq b) = F_X(b) - F_X(a)\)
        \item Via les fonctions de densité/de masse :
        \begin{itemize}
            \item Pour des variables continues : \(\distribution(a < X \leq b) = \int_{a}^b f_X(x)\,dx\)
            \item Pour des variables discrètes : \(\distribution(a < X \leq b) = \sum_{x > a}^{x = b} \distribution(X = x)\)
        \end{itemize}
    \end{itemize}