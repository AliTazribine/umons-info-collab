\documentclass[usenames,dvipsnames]{article}

\usepackage{polyglossia}
\setdefaultlanguage{french}

\usepackage{amsmath}
\usepackage{amsthm}
\usepackage{amssymb}
\usepackage{mathtools}

\newcommand{\N}{\mathbb{N}\xspace}

\renewcommand{\epsilon}{\varepsilon}
\renewcommand{\phi}{\varphi}

\allowdisplaybreaks % Autorise les environnements à faire un page break

% J'ai suivi les recommandations de http://mirror.koddos.net/CTAN/macros/latex/required/amscls/doc/amsthdoc.pdf
\theoremstyle{plain}
\newtheorem{thm}{Théorème}[section]
\newtheorem{theorem}[thm]{Théorème} % Parce qu'à chaque fois, je me trompe
\newtheorem{lemme}{Lemme}[section]
\newtheorem{corollaire}{Corollaire}[section]

\theoremstyle{definition}
\newtheorem{definition}{Définition}[section]
\newtheorem{propriete}{Propriété}[section]
\newtheorem{exemple}{Exemple}[section]

\theoremstyle{remark}
\newtheorem{note}{Note}[section]
\newtheorem{rem}{Remarque}[section]
\newtheorem{remarque}[rem]{Remarque}
\newtheorem{illustration}{Illustration}[section]

\newenvironment{demonstration}{\begin{proof}[\textnormal{\textbf{Preuve}}]}{\end{proof}}
\newenvironment{preuve}{\begin{demonstration}}{\end{demonstration}\ignorespacesafterend}

% Traductions
\newcommand{\thmautorefname}{Théorème}
\renewcommand{\theoremautorefname}{Théorème}
\newcommand{\proprieteautorefname}{Propriété}
\newcommand{\exempleautorefname}{Exemple}
\newcommand{\lemmeautorefname}{Lemme}
\newcommand{\definitionautorefname}{Définition}
\newcommand{\algorithmautorefname}{Algorithme}
\usepackage{algorithm}
\usepackage{algpseudocode}

\algblockdefx[doWhile]{DoWhile}{EndDoWhile}{\algorithmicdo}[1]{\algorithmicwhile~#1}
\algblockdefx[doUntil]{DoUntil}{EndDoUntil}{\algorithmicdo}[1]{\algorithmicuntil~#1}

\addto\captionsfrench{
    \floatname{algorithm}{Algorithme}
    \algrenewcommand\algorithmicrequire{\textbf{Entrée(s)} :}
    \algrenewcommand\algorithmicensure{\textbf{Sortie(s)} :}
    \algrenewcommand\algorithmicdo{\textbf{faire}}
    \algrenewcommand\algorithmicwhile{\textbf{tant que}}
    \algrenewcommand\algorithmicuntil{\textbf{jusqu'à ce que}}
    \algrenewcommand\algorithmicforall{\textbf{pour chaque}}
    \algrenewcommand\algorithmicfor{\textbf{pour}}
    \algrenewcommand\algorithmicend{\textbf{fin}}
    \algrenewcommand\algorithmicreturn{\textbf{retourner}}
}

\usepackage[backend = biber, style = alphabetic]{biblatex}

\addbibresource{preambule/biblio/Numerical_Analysis.bibtex}

% Pour l'index
\usepackage{makeidx}
\makeindex{}
% Pour la liste des abbréviations
\usepackage[intoc]{nomencl}
\makenomenclature
\addto\captionsfrench{
    \renewcommand{\nomname}{Liste des abbréviations}
}

\usepackage{xspace}
\usepackage{fullpage}

\usepackage{xcolor}
\usepackage{listings}
\usepackage{lstautogobble}
\lstset {
    language=R,
    commentstyle=\color{Green},
    stringstyle=\color{Orange},
    autogobble = true
}

\usepackage[hidelinks]{hyperref}
\usepackage[nameinlink]{cleveref}

\newcommand{\transfoZ}{\xLeftrightarrow{Z}}
\newcommand{\fourier}{\xLeftrightarrow{F}}
\newcommand{\TFTD}{\xLeftrightarrow{TFTD}}

\newcommand{\NTFD}{N_{TFD}}

\newcommand{\sumInfty}[1]{\sum\limits_{#1=-\infty}^{+\infty}}
\newcommand{\intInfty}{\int_{-\infty}^{+\infty}}

\title{Formulaire/Synthèse de \textit{Big Data Analytics I}}
\author{Gaëtan Staquet}
\date{Année académique 2018--2019}

\begin{document}
    \maketitle

    \begin{abstract}
        Sauf mentions contraires, toutes les informations données dans ce document proviennent des slides du cours de \textit{Big Data Analytics I} donné par M. Souhaib Ben Taieb en l'année académique 2018-2019.
    \end{abstract}

    \tableofcontents

    \clearpage
    \section{Algèbre linéaire}
    Nous donnons ici quelques formules qui pourraient être utiles. Des définitions complémentaires peuvent être trouvées dans le reste du document (comme la définition des valeurs propres). Les définitions et formules proviennent principalement de Wikipédia.

    \subsection{Signe de la dérivée seconde et maximum/minimum}
        Un point \(x\) d'une fonction \(f\) est un \textit{minimum} si \(f'(x) = 0\) et \(f''(x) > 0\).

        Un point \(x\) d'une fonction \(f\) est un \textit{maximum} si \(f'(x) = 0\) et \(f''(x) < 0\).

    \subsection{Vecteurs}
        \begin{definition}
            Soit \(q \geq 1\). La \textit{q-norme}\index{Norme} d'un vecteur \(x = (x_1, \dots, x_p)\) est
            \[
                \norm{x}_q = \left(\sum_{j=1}^p |x_j|^q\right)^{\frac{1}{q}}
            \]

            Quelques normes particulières :
            \begin{itemize}
                \item \(q = 1\): norme \(L_1\) (terme pénalisant de la régression Lasso)
                \item \(q = 2\): norme \(L_2\), norme euclidienne (terme pénalisant de la régression Ridge)
                \item \(q = \infty\): norme \(L_\infty\), norme uniforme :
                \[
                    \norm{x}_\infty = \max\{|x_1|, \dots, |x_p|\}
                \]
            \end{itemize}
        \end{definition}

        \begin{definition}
            Le \textit{produit scalaire} de deux vecteurs \(x, y \in \R^n\) est :
            \begin{align*}
                xy &= \norm{x}_2 \norm{y}_2 \cos(\theta) & \theta \text{ est l'angle entre \(x\) et \(y\)} \\
                &= \sum_{i=1}^n x_i y_i
            \end{align*}

            On peut aussi voir le produit scalaire comme un produit de deux matrices. Si on considère les deux vecteurs comme étant des vecteurs colonnes, le produit scalaire est donné par \(xy^T\).

            Le produit scalaire est commutatif et distributif.
        \end{definition}

        \begin{definition}
            Un vecteur est dit \textit{unitaire} si sa 2-norme vaut 1.

            Deux vecteurs \(x\) et \(y\) sont \textit{orthogonaux} si leur produit scalaire est nul, c'est-à-dire si \(u v = 0\).

            Deux vecteurs sont \textit{orthonormaux} s'ils sont unitaires et orthogonaux.
        \end{definition}

        \begin{definition}
            Deux vecteurs \(x, y \in \R^n\) sont \textit{linéairement dépendants} si l'un peut être exprimé comme une combinaison linéaire de l'autre, c'est-à-dire s'il existe deux scalaires \(a_1, a_2\) (non tous deux nuls, c'est-à-dire \(a_1 \not= 0 \lor a_2 \not= 0\)) tels que \(a_1 x + a_2 y = 0\) (où \(0\) est le vecteur nul).

            Deux vecteurs sont \textit{linéairement indépendants} s'ils ne sont pas linéairement dépendants.
        \end{definition}

    \subsection{Matrices}
        \begin{definition}
            Le \textit{rang} d'une matrice est le nombre de colonnes linéairement indépendantes.
        \end{definition}

        \begin{definition}
            La \textit{trace} d'une matrice carrée \(A \in \R^{n\times n}\) est 
            \[
                \trace(A) = \sum_{i=1}^n a_{ii}
            \]
        \end{definition}

        \begin{propriete}
            Soient \(A \in \R^{n \times p}, B \in \R^{p\times n}\). On a:
            \[
                \trace(AB) = \trace(BA)
            \]
        \end{propriete}

        \begin{definition}
            Le \textit{produit} de deux matrices \(A \in \R^{n \times m}, B \in \R^{m \times p}\) est la matrice \(C \in \R^{n \times p}\) telle que
            \[
                \forall i \in \{1, \dots, n\}, \forall j \in \{1, \dots, p\}, c_{ij} = \sum_{k=1}^m a_{ik} b_{kj}
            \]

            Le produit matriciel n'est pas commutatif mais est distributif et associatif.

            On a :
            \[
                (AB)^T = B^T A^T
            \]
        \end{definition}
        
        \subsection{Inverse}
            \begin{definition}
                Une matrice \(A \in \R^{n \times n}\) est dite \textit{inversible} (ou \textit{non-singulière}) si et seulement si (les conditions suivantes sont équivalentes) :
                \begin{itemize}
                    \item Son déterminant est non-nul.
                    \item Il existe une unique matrice \(B \in \R^{n \times n}\) telle que \(AB = BA = \identity_n\). \(B\) est appelé l'\textit{inverse} de \(A\) et est souvent notée \(A^{-1}\)
                    \item 0 n'est pas une valeur propre de \(A\)
                    \item Le rang de la matrice est \(n\)
                \end{itemize}
            \end{definition}

            \begin{propriete}
                On a les propriétés suivantes :
                \begin{align*}
                    (A^{-1})^{-1} &= A\\
                    \forall k \in \R\setminus\{0\}, (kA)^{-1} &= k^{-1}A^{-1}\\
                    (A^T)^{-1} &= (A^{-1})^T\\
                    \det(A^{-1}) &= (\det(A))^{-1}\\
                    (AB)^{-1} &= (B^{-1} A^{-1})
                \end{align*}
            \end{propriete}

            \subsubsection{Calculs}
                Nous donnons ici quelques formules pour calculer l'inverse d'une matrice (les formules qui semblent les plus utiles vu le reste du cours).

                \paragraph{Décomposition en valeurs propres}
                    Si une matrice peut être décomposée en valeurs propres, c'est-à-dire si on peut écrire \(A = Q \Lambda Q^T\), alors \(A\) est inversible et son inverse est donnée par :
                    \[
                        A^{-1} = Q \Lambda^{-1} Q^{-1}
                    \]
                    L'inverse de \(\Lambda\) est simple à calculer car \(\Lambda\) est une matrice diagonale.

                \paragraph{Solution analytique}
                    \begin{definition}
                        Le \textit{cofacteur} d'indice \(i,j\) de \(A\in\R^{n\times n}\) est :
                        \[
                            C_{ij} = (-1)^{i+j} \det(A_{i,j})
                        \]
                        où \(A_{i,j}\) est la sous-matrice carrée de taile \(n-1\) déduite de \(A\) en supprimant la \(i\)\ieme{} ligne et la \(j\)\ieme{} colonne.
                    \end{definition}

                    Soit \(C\), la matrice des cofacteurs de \(A\). Alors,
                    \begin{align*}
                        A^{-1} &= \frac{1}{\det(A)} C^T\\
                        (A^{-1})_{ij} &= \frac{1}{\det(A)} (C_{ji})
                    \end{align*}

                \paragraph{Matrice \(2 \times 2\)}
                    En utilisant les cofacteurs, on obtient :
                    \[
                        A^{-1} = \begin{pmatrix}
                            a & b\\
                            c & d
                        \end{pmatrix}^{-1} = \frac{1}{ad - bc} \begin{bmatrix}
                            d & -b\\
                            -c & a
                        \end{bmatrix}
                    \]
    \section{Statistiques}
    \subsection{Probabilités}
    \begin{definition}[Événement]
        Le \textit{sample space} \(\Omega\) est l'ensemble des valeurs élémentaires possibles. Par exemple, pour un lancer de pièces, les valeurs possibles sont \(\{heads, tails\}\).

        Un \textit{événement} est un sous-ensemble \(A \subseteq \Omega\). On dit qu'un événement \(A\) a lieu si le résultat d'une expérience est dans \(A\).
    \end{definition}

    \begin{definition}[Distributions]
        Une \textit{distribution de probabilités} est une fonction \(\distribution : A \to \R\). Cette fonction doit satisfaire certains axiomes :
        \begin{enumerate}
            \item Non-négative : \(\forall A \subseteq \Omega, \distribution(A) \geq 0\).
            \item Unité de \(\Omega\) : \(\distribution(\Omega) = 1\).
            \item Countable additivity : Pour une suite \(A_1, A_2, \dots\) d'ensembles disjoints, on a :
            \[
                \distribution(\bigcup_{i = 1}^\infty A_i) = \sum_{i=1}^\infty \distribution(A_i)
            \]
        \end{enumerate}

        Avec ces axiomes, il est possible de montrer les propriétés suivantes :
        \begin{itemize}
            \item \(\distribution(\emptyset) = 0\)
            \item \(\forall A, B \subseteq \Omega, A \subset B \implies \distribution(A) \leq \distribution(B)\)
            \item \(\forall A \subseteq \Omega, 0 \leq \distribution(A) \leq 1\)
            \item \(\forall A \subseteq \Omega, \distribution(\complementary{A}) = 1 - \distribution(A)\)
            \item \(\forall A, B \subseteq \Omega, \distribution(A \cup B) = \distribution(A) + \distribution(B) - \distribution(A \cap B) \Leftrightarrow \distribution(A \cap B) = \distribution (A) + \distribution(B) - \distribution(A \cup B)\)
            \item \(\implies \distribution(A \cap B) \geq \distribution(A) + \distribution(B) - 1\) car \(\distribution(A \cup B) \leq 1\)
        \end{itemize}
    \end{definition}

    \subsubsection{Probabilités conditionnelles et événéments indépendants}
        \begin{definition}
            Soient \(A, B \subseteq \Omega\), si \(\distribution(B) > 0\), alors la \textit{probabilité conditionnelle} de \(A\) étant donnée \(B\) est :
            \[
                \distribution(A | B) = \frac{\distribution(A \cup B)}{\distribution(B)}
            \]
        \end{definition}

        \begin{definition}
            Soient \(A, B \subseteq \Omega\), \(A\) et \(B\) sont dits \textit{indépendants} si 
            \[
                \distribution(A \cap B) = \distribution(A)\distribution(B)
            \]
            ou si
            \[
                \distribution(A | B) = \distribution(A)   
            \]

            Un ensemble d'événéments \(A_j (j \in I)\) sont dits \textit{mutuellement indépendants} si
            \[
                \forall J \subseteq I, \distribution\left(\bigcap_{j\in J}\right) = \prod_{j\in J} \distribution(A_j)
            \]
        \end{definition}

    \subsubsection{Règle de Bayes}
        \begin{theorem}[Loi de la probabilité totale]
            Soit \(A_1, \dots, A_k\) une partition de \(\Omega\). Alors,
            \[
                \forall B \subseteq \Omega, \distribution(B) = \sum_{i=1}^k \distribution(B | A_i) \distribution(A_i)
            \]
        \end{theorem}
        \begin{proof}
            On a que \(A_i \cap B (i = 1, \dots, k)\) forme une partition de \(B\) et \(\distribution(A_i \cap B) = \distribution(B | A_i) \distribution(A_i)\).
        \end{proof}

        \begin{theorem}[Bayes]
            Soit \(A_1, \dots, A_k\) une partition de \(\Omega\). Alors,
            \[
                \distribution(A_i | B) = \frac{\distribution(B | A_i) \distribution(A_i)}{\sum_{i=1}^k \distribution(B | A_i)\distribution(A_i)}
            \]
        \end{theorem}
    \subsection{Variables aléatoires et distributions}
    \begin{definition}
        Une \textit{variable aléatoire} est une fonction \(\Omega \to \R\). 
        
        Une façon de voir une variable aléatoire est de penser à un mapping entre une distribution sur \(\Omega\) et une distribution sur les réels (c'est-à-dire l'ensemble des valeurs de la variable aléatoire).
        Formellement, pour une variable \(X\) et un sous-ensemble \(A \in \R\) :
        \[
            \distribution_X (X \in A) = \distribution(\{\omega \in \Omega : X(\omega) \in A\})
        \]
    \end{definition}

    \begin{definition}
        Chaque variable aléatoire est associée à une \textit{fonction de distribution cumulative} (notée CDF) :
        \[
            \forall x, F_X(x) = \distribution_X(X \leq x)
        \]

        Une fonction \(F\) est une CDF si et seulement si :
        \begin{enumerate}
            \item \(\lim_{x \to -\infty} F(x) = 0\) et \(\lim_{x \to +\infty} F(x) = 1\)
            \item La fonction n'est pas décroissante en \(x\)
            \item La CDF est continue à droite, c'est-à-dire, \(\forall x_0 \in \R, \lim_{x \to x_0^+} F(x) = F(x_0)\)
        \end{enumerate}

        \(X\) est une variable continue si sa CDF est une fonction continue et est une variables discrète si sa CDF est une fonction discrète.

        Deux variables aléatoires \(X\) et \(Y\) sont identiquement distribués si \(\forall A, \distribution_X(X \in A) = \distribution_Y(Y \in A)\) (ne veut pas dire que \(X\) et \(Y\) sont égaux).

        Deux variables \(X\) et \(Y\) sont identiquement distribués si et seulement si leur CDF sont égaux, c'est-à-dire, \(\forall x, F_x(x) = F_y(x)\)
    \end{definition}

    \begin{remarque}[Notations]
        \(F_x(x)\) indique une CDF tandis que \(f_X(x)\) indique une pdf/pmf.
    \end{remarque}

    \begin{definition}
        Pour une variable discrète, sa \textit{fonction de masse} (notée \textit{PMF}) :
        \[
            f_X(x) = P_X(X = x)
        \]

        Pour une variable continue, sa \textit{densité de probabilité} (notée \textit{PDF}) \(f_X\) est la fonction qui satisfait :
        \[
            \forall x, F_X(x) = \int_{-\infty}^{x} f_X(t)\,dt
        \]

        Une fonction \(f_X(x)\) est une pdf/pmf si et seulement si :
        \begin{enumerate}
            \item \(\forall x, f_X(x) \geq 0\)
            \item \(\sum_x f_X(x) = 1\) (pour une pmf) ou \(\int_{-\infty}^{+\infty} f_X(x)\,dx = 1\) (pour une pdf)
        \end{enumerate}
    \end{definition}

    Pour trouver la probabilité qu'une variable aléatoire atterrisse dans un intervalle, il y a deux façons :
    \begin{itemize}
        \item Via les fonctions de distribution : \(\distribution(a < X \leq b) = F_X(b) - F_X(a)\)
        \item Via les fonctions de densité/de masse :
        \begin{itemize}
            \item Pour des variables continues : \(\distribution(a < X \leq b) = \int_{a}^b f_X(x)\,dx\)
            \item Pour des variables discrètes : \(\distribution(a < X \leq b) = \sum_{x > a}^{x = b} \distribution(X = x)\)
        \end{itemize}
    \end{itemize}
    \subsection{Quelques distributions}
    \subsubsection{Distributions discrètes}
        \paragraph{Distribution uniforme (discrète)}
            Sur \(k\) catégories \(\{x_1, x_2, \dots, x_k\}\), la distribution uniforme discrète est :
            \[
                \forall x \in \{x_1, x_2, \dots, x_k\}, p_X(x) = \frac{1}{k}
            \]

        \paragraph{Distribution de Bernoulli}
            Typiquement, la distribution d'un lancer de pièces, c'est-à-dire que \(x \in \{0, 1\}\). On a \(p\) qui donne la probabilité d'avoir \(1\). La pmf de Bernoulli est :
            \[
                \forall x \in \{0, 1\}, p_X(x) = p^x (1-p)^{1-x}
            \]

            Cette distribution est notée \(\bernoulli(p)\).

            Pour \(X \sim \bernoulli(p), \expectation[X] = p * 1 + (1 - p) * 0 = p\) et \(\variance[X] = p (1 - p)\).

        \paragraph{Distribution binomiale}
            Typiquement, la distribution du nombre de heads dans \(n\) lancers de pièces :
            \[
                \forall x, p_X(x) = \begin{cases}
                    \binom{n}{x} p^x (1 - p)^{n-x} & \text{si } x \in \{0, 1, \dots, n\}\\
                    0 & \text{sinon}
                \end{cases}
            \]

            Cette distribution est notée \(\binomial(n, p)\).

        \paragraph{Distribution géométrique}
            Typiquement, la distribution du nombre de lancers pour voir une face. Sa pmf :
            \[
                \forall x \in \{1, 2, \dots\}, p_X(x) = p(1-p)^{x_1}
            \]

            Cette distribution est notée \(\geometric(p)\).

        \paragraph{Distribution de Poisson}
            Une distribution de Poisson de moyenne \(\lambda\) a comme pmf :
            \[
                \forall x \in \{0, 1, \dots\}, p_X(x) = \frac{\lambda^x e^{-\lambda}}{x!}
            \]

            Cette distribution est notée \(\poisson(\lambda)\).

    \subsubsection{Distributions continues}
        \paragraph{Distribution uniforme (continue)}
            Sur \([a, b]\), sa pdf est :
            \[
                \forall x, p_X(x) = \begin{cases}
                    \frac{1}{b - a} & \text{si } x \in [a, b]\\
                    0 & \text{sinon}
                \end{cases}
            \]

            Cette distribution est notée \(\uniform[a, b]\).

        \paragraph{Distribution gaussienne}
            Cette distribution a une moyenne \(\mu\) et une variance \(\sigma^2\). Sa pdf est :
            \[
                \forall x, p_X(x) = \frac{1}{\sqrt{2\pi}\sigma} e^{-\frac{1}{2\sigma^2}(x - \mu)^2}
            \]

            Cette distribution est notée \(\gaussian(\mu, \sigma^2)\).
    \subsection{Moments, Espérance, Variance et Covariance}
    \begin{definition}
        L'\textit{espérance} ou \textit{moyenne} ou \textit{premier moment} d'une variable aléatoire \(X\) est définie par :
        \[
            \expectation[X] = \int x\,dF_X(x) = \int xf_X(x)\,dx \text{\hspace{1em}ou\hspace{1em}}  \sum_x xf_X(x)
        \]

        Si \(\expectation[X] = \infty\), on dit que l'espérance n'existe pas. % il n'y a pas d'espoir dans ce monde

        Quand le nombre d'expériences \(n\) est très grand, \(\expectation[n] \approx \frac{1}{n} \sum_{i=1}^n X_i\).
    \end{definition}

    Pour une variable aléatoire \(Y = r(X)\) (donc une transformation de \(X\)), l'espérance est donnée par la \textit{règle du statisticien fainéant} :
    \[
        \expectation[Y] = \expectation[r(X)] = \int_x r(x)\,dF_X(x)
    \]

    \begin{theorem}
        L'espérance est linéaire. En d'autres termes, pour une collection de variables aléatoires \(X_1, \dots, X_n\) et des constantes \(a_1, \dots, a_N\) :
        \[
            \expectation\left[\sum_i a_iX_i\right] = \sum_i a_i \expectation[X_i]
        \]
    \end{theorem}

    \begin{definition}
        Pour une variable aléatoire \(X\), son \textit{\(k\ieme{}\) moment} est :
        \[
            \mu_k = \expectation[X^k]
        \]

        On note généralement la moyenne par \(\mu\) au lieu de \(\mu_1\).

        Les \textit{moments centrés} sont définis par :
        \[
            \alpha_k = \expectation\left[(X - \mu)^k\right]    
        \]
    \end{definition}

    \begin{definition}
        Le second moment centré d'une variable aléatoire \(X\) est appelée sa \textit{variance}. On la note \(\sigma_X^2\). Sa racine carrée est l'\textit{écart-type}.

        On a que :
        \[
            \sigma^2_X = \expectation\left[(X - \mu)^2\right] = \expectation[X^2 + \mu^2 - 2\mu X] = \expectation[X^2] - \mu^2 \text{\hspace{2em} Par déf des moments et linéarité de \(\expectation\)}
        \]

        Pour des constantes \(a, b\), on a que :
        \[
            \sigma_{aX+b}^2 = a^2\sigma_X^2
        \]
    \end{definition}

    \begin{definition}
        La \textit{covariance} entre deux variables aléatoires \(X\) et \(Y\) est définie par :
        \[
            \covariance[X, Y] = \expectation[(X - \mu_X)(Y - \mu_Y)] = \expectation[XY] - \expectation[X]\expectation[Y]
        \]

        La covariance de deux variables aléatoires indépendantes est nulle.

        La \textit{corrélation} entre deux variables aléatoires \(X\) et \(Y\) est la forme standardisée de la covariance :
        \[
            \correlation[X, Y] = \frac{\covariance[X, Y]}{\sigma_X \sigma_Y}
        \]

        On a que \(-1 \leq \correlation[X, Y] \leq 1\).
    \end{definition}

    \begin{theorem}
        Pour des variables aléatoires \(X_1, \dots, X_n\) et des constantes \(a_1, \dots, a_n\), on a :
        \[
            \variance\left[\sum_{i=1}^n a_i X_i\right] = \sum_{i=1}^n \sum_{j=1}^n a_i a_j \covariance[X_i, X_j]
        \]
    \end{theorem}
    \begin{proof}
        On peut utiliser le résultat suivant :
        \[
            \left(\sum_{i=1}^n x_i\right)^2 = \sum_{i=1}^n \sum_{j=1}^n x_i x_j
        \]

        On a :
        \begin{align*}
            \variance\left[\sum_{i=1}^n a_i X_i\right] &= \expectation\left[\left(\sum_{i=1}^n a_i X_i - \expectation\left[\sum_{i=1}^n a_i X_i\right]\right)^2\right] & \text{Par définition de \(\variance\)}\\
            &= \expectation\left[\left(\sum_{i=1}^n a_i X_i - \sum_{i=1}^n a_i \expectation[X_i]\right)^2\right] & \text{Par linéarité de \(\expectation\)}\\
            &= \expectation\left[\left(\sum_{i=1}^n \left(a_i (X_i - \expectation[X_i])\right)\right)^2\right]\\
            &= \expectation\left[\sum_{i=1}^n \sum_{j=1}^n \left(a_i (X_i - \expectation[X_i])\right)\left(a_j \left(X_j - \expectation[X_j]\right)\right)\right] & \text{Par le résultat donné}\\
            &= \expectation\left[\sum_{i=1}^n \sum_{j=1}^n a_ia_j (X_i - \expectation[X_i])(X_j - \expectation[X_j])\right]\\
            &= \sum_{i=1}^n \sum_{j=1}^n a_i a_j \expectation\left[(X_i - \expectation[X_i])(X_j - \expectation[X_j])\right] & \text{Par linéarité de \(\expectation\)}\\
            &= \sum_{i=1}^n \sum_{j=1}^n a_i a_j \covariance[X_i, X_j] & \text{Par définition de \(\covariance\)}
        \end{align*}
    \end{proof}

    \begin{remarque}
        On peut encore simplifier cette expression. En effet, par le fait que \(\covariance[X_i, X_j] = \covariance[X_j, X_i]\) et que \(\covariance[X_i, X_i] = \variance[X_i]\), on obtient :
        \[
            \variance\left[\sum_{i=1}^n X_i\right] = \sum_{i=1}^n \variance[X_i] + 2\sum_{i=1}^n \sum_{j=i+1}^n \covariance[X_i, X_j]
        \]
    \end{remarque}

    \begin{theorem}
        Pour \(n\) variables aléatoires iid \(X_1, \dots, X_n\) :
        \[
            \variance\left[\frac{1}{n} \sum_{i=1}^n X_i\right] = \frac{1}{n^2} \sum_{i=1}^n \variance[X_i] = \frac{\sigma_X^2}{n}
        \]
    \end{theorem}
    \begin{proof}
        \begin{align*}
            \variance\left[\frac{1}{n} \sum_{i=1}^n X_i\right] &= \sum_{i=1}^n \sum_{j=1}^n \frac{1}{n^2} \covariance[X_i, X_j] & \text{Théorème précédent}\\
            &= \frac{1}{n^2} \sum_{i=1}^n \sum_{j=1}^n \covariance[X_i, X_j]\\
            &= \frac{1}{n^2} \left(\sum_{i=1}^n \variance[X_i] + 2\sum_{i=i+1}^n\sum_{j=1}^n \covariance[X_i, X_j]\right) & \text{Par la remarque précédente}\\
            &= \frac{1}{n^2} \sum_{i=1}^n \variance[X_i] & \text{Car \(X_i\) et \(X_j\) sont indépendants}
        \end{align*}
    \end{proof}

    \subsubsection{Espérance conditionnelle}
        \begin{definition}
            Soient deux variables aléatoires \(X\) et \(Y\). On veut calculer la valeur moyenne de \(Y\) quand \(X = x\). L'\textit{espérance conditionnelle} est :
            \[
                \expectation[Y | X = x] = \sum_y y f_{Y|X}(y | x) \text{\hspace{2em}ou\hspace{2em}} \expectation[Y | X = x] = \int_y y f_{Y|X}(y|x)\,dy
            \]

            L'espérance conditionnelle est une fonction en \(X\) (contrairement à l'espérance d'une variable aléatoire).
        \end{definition}

        \begin{theorem}[Indépendance]
            Si deux variables aléatoires \(X\) et \(Y\) sont indépendantes alors
            \[
                \expectation[Y | X = x] = \expectation[Y]
            \]

            En général, l'implication dans l'autre sens est fausse (des variables aléatoires dépendantes peuvent satisfaire l'expression).
        \end{theorem}

        \begin{definition}[Loi de l'espérance totale]
            Aussi appelée la \textit{tower property}, la \textit{loi de l'espérance totale} :
            \[
                \expectation_X[\expectation_{Y|X}[Y|X]] = \expectation_Y[Y]
            \]
        \end{definition}

    \subsubsection{Variance conditionnelle}
        \begin{definition}
            La \textit{variance conditionnelle} est :
            \[
                \variance[Y | X = x] = \expectation[(Y - \expectation[Y | X = x])^2 | X = x]
            \]
        \end{definition}

        \begin{definition}[Loi de la variance totale]
            La \textit{loi de la variance totale} est :
            \[
                \variance[Y] = \expectation[\variance[Y | X]] + \variance[\expectation[Y | X]]
            \]
        \end{definition}
    \subsection{Inférence statistique}
    Le but de l'inférence statistique est d'inférer des choses sur \(F\) dans \(X_1, \dots, X_n \sim F\). Il existe deux catégories de modèles statistiques :
    \begin{itemize}
        \item Modèle paramétrique : l'ensemble des distributions \(\mathcal{F}\) peut être décrit par un nombre fini de paramètres. Par exemple, un modèle gaussien peut être décrit par sa moyenne et sa variance et un modèle Bernoulli peut être décrit par son paramètre \(p\).
        \item Modèle non-paramétrique : l'ensemble des distributions \(\mathcal{F}\) ne peut pas être décrit par un nombre fini de paramètres. Par exemple, estimer directement la CDF ou la densité.
    \end{itemize}

    Pour le reste de cette section, on suppose que l'échantillon a été généré par un modèle paramétrique. On veut donc estimer les paramètres.

    \subsubsection{Méthode des moments}
        Supposons qu'il y a \(k\) paramètres à estimer \(\theta = (\theta_1, \dots, \theta_K)\). On peut estimer \(\theta\) en trouvant \(k\) moments. Soient
        \[
            m_1 = \frac{1}{n} \sum_{i=1}^n X_i, m_2 = \frac{1}{n} \sum_{i=1}^n X_i^2, \dots, m_k = \frac{1}{n} \sum_{i=1}^n X_i^k
        \]

        Soit \(\mu_i(\theta) = \int x^i p_\theta(x)\,dx\) le moment du \(i\ieme{}\) individu. La \textit{méthode des moments} consiste à résoudre le système suivant :
        \[
            \begin{cases}
                m_1 &= \mu_1(\theta_1, \dots, \theta_k)\\
                &\vdots\\
                m_k &= \mu_k(\theta_1, \dots, \theta_k)
            \end{cases}
        \]

    \subsubsection{Maximum Likelihood Estimation}
        Supposons que \(X_1, \dots, X_n \sim p_\theta\) avec \(p_\theta\) la pmf ou la pdf.

        \begin{definition}
            La fonction \textit{likelihood}\index{Likelihood} est définie par :
            \[
                L(\theta) \equiv L(\theta, X_1, \dots, X_n) = \prod_{i=1}^n p_\theta(X_i)
            \]

            La fonction \textit{log-likelihood} est définie par :
            \[
                l(\theta) \equiv l(\theta, X_1, \dots, X_n) = \log(L(\theta))
            \]

            Le \textit{Maximum Likelihood Estimator} (noté \textit{MLE}\index{MLE})\nomenclature{MLE}{Maximum Likelihood Estimator (ou Estimation)} est la valeur de \(\theta\) qui maximise \(L(\theta)\) (et \(l(\theta)\) car le \(\log\) est une fonction croissante). Cette valeur est notée \(\estimation{\theta}\) :
            \[
                \estimation{\theta} = \argmax_\theta L(\theta) = \argmax_\theta l(\theta)
            \]
        \end{definition}

        Typiquement, on calcule le MLE en dérivant partiellement \(l(\theta)\). En d'autres termes, on résoud le système d'équations :
        \[
            \forall i \in \{1, \dots, k\}, \frac{\partial}{\partial \theta_i} l(\theta) = 0
        \]
    \section{Apprentissage supervisé}
    \subsection{Éléments communs}
        \begin{definition}
            Les \textit{variables d'entrée}\index{Variable!d'entrée} (\textit{input variables}\index{Input variable|see {Variable, d'entrée}}), aussi appelées \textit{predictors}\index{Predictor|see {Variable, d'entrée}}, \textit{variables indépendantes}\index{Variable!indépendante|see {Variable, d'entrée}}, \textit{features}\index{Feature|see {Variable, d'entrée}} ou simplement \textit{variables}, sont représentées par le symbole \(X\). S'il y a \(p\) variables différentes, on écrit :
            \[X = (X_1, X_2, \dots, X_p)\]
        \end{definition}

        \begin{remarque}
            \(X\) est une variable aléatoire. Une réalisation de cette variable (comme, par exemple, une instance dans un jeu de données) est notée \(x\).
        \end{remarque}

        \begin{definition}
            La \textit{variable de sortie}\index{Variable!de sortie} (\textit{output variable}\index{Output variable|see {Variable, de sortie}}), aussi appelée \textit{réponse}\index{Réponse|see {Variable, de sortie}} ou \textit{variable dépendante}\index{Variable!dépendante|see {Variable, de sortie}} est souvent notée \(Y\).
        \end{definition}

        On suppose qu'il existe une relation aléatoire entre \(X\) et \(Y\), c'est-à-dire :
        \[\distribution(X, Y) = \distribution(X)\distribution(Y|X) \not= \distribution(X)\distribution(Y)\]

        \begin{definition}
            Les \textit{données}\index{Données} :
            \[
                \data = \{(x_1, y_1), (x_2, y_2), \dots, (x_n, y_n)\} = \{(x_i, y_i\}^n_{i=1}
            \]
            où \(x_i = (x_{i,1}, \dots, x_{i,p})\), c'est-à-dire que \(x_i\) est le tuple des variables pour la i\ieme{} instance (ligne) dans les données.

            On peut voir les données comme un échantillon i.i.d. de la véritable distribution, c'est-à-dire :
            \[(x_i, y_i) \stackrel{i.i.d.}{\sim} \distribution(X, Y)\]
            \nomenclature{i.i.d.}{Indépendant et identiquement distribué}
        \end{definition}

        On peut décrire tous les problèmes d'apprentissage supervisé avec les éléments suivants :
        \begin{itemize}
            \item Les \textit{variables d'entrée}\index{Variable!d'entrée} \(X\) et la \textit{variable de sortie}\index{Variable!de sortie} \(Y\);
            \item Les \textit{données}\index{Données} \(\data = \{(x_i, y_i)_{i=1}^n\}\);
            \item La \textit{fonction cible}\index{Fonction!cible}. Par exemple, la distribution jointe \(\distribution(X, Y)\), l'espérance conditionnelle \(\expectation[Y|X]\), la distribution conditionnelle \(\distribution(Y|X)\), etc.;
            \item L'\textit{ensemble d'hypothèses}\index{Ensembles d'hypothèses} \(\hypo\) contient toutes les hypothèses à considérer (les fonctions à tester). Souvent, cet ensemble est défini implicitement; et
            \item La \textit{fonction de perte}\index{Fonction!de perte} (ou \textit{fonction de coût}\index{Fonction!de coût|see {Fonction, de perte}}) \(L(y, \estimation{y})\) est une fonction \(\R \times \R \to \R^+\). Par exemple, \(L(y, \estimation{y}) = (y - \estimation{y})^2\).
        \end{itemize}
        L'\textit{algorithme d'apprentissage}\index{Algorithme!d'apprentissage} choisi la meilleure hypothèse de \(\hypo\) en utilisant les données \(\data\) et la fonction de perte \(L\).

        \subsubsection{Erreurs}
            \begin{definition}
                L'\textit{erreur de test}\index{Erreur!de test} (ou \textit{erreur out-of-sample}\index{Erreur!out-of-sample|see {Erreur, de test}}) de l'hypothèse \(\fEstimated\) permet de tester la performance de l'hypothèse sur de nouvelles données et est défini par :
                \[
                    \errorOut(\fEstimated) = \expectation[L(Y, \fEstimated(X))]
                \]
            \end{definition}

            On veut sélectionner la meilleure hypothèse :
            \[
                \fEstimated = \argmin_{h \in \hypo} \errorOut(\widehat{h}_\data)
            \]

            Cependant, on ne sait pas calculer \(\errorOut(\fEstimated)\) car on ne connait pas \(\distribution(Y, X)\).

        \subsubsection{Biais et variance}
            \begin{definition}
                Le \textit{biais}\index{Biais} est l'erreur introduite en modélisant un problème compliqué par un problème plus simple.

                La \textit{variance}\index{Variance} permet de mesurer à quel point le modèle construit serait différent si les données d'entraînement étaient différentes.
            \end{definition}

            En général, une méthode plus flexible implique un biais plus petit et une variance plus grande (car la méthode colle mieux aux données d'entraînement).

            \begin{remarque}
                La taille de l'ensemble d'entraînement a un impact sur la variance. Un plus grand ensemble réduit la variance !
            \end{remarque}

    \section{Régression avancée}
    Rappel : solution optimale de la régression linéaire donnée par OLS : \(\betaLS = (X^TX)^{-1} X^Ty\)

    Cette solution peut avoir des problèmes en haute dimension parce qu'il peut arriver que certaines colonnes de \(X\) ne sont pas linéairement indépendantes. Par conséquent, \(X\) n'est pas `full rank'. Dans ce cas, \(X^T X\) est singulière (non-inversible) et les coefficients OLS n'ont pas une unique valeur.

    Pour rappel, le biais de \(\betaLS\) est \(\expectation[\betaLS] - \beta^* = 0\) et la variance est \(\variance[\betaLS] = \sigma^2 (X^T X)^{-1}\). Si \(X\) a des colonnes orthonormales :
    \[
        \trace(\variance[\betaLS]) = \sigma^2 p
    \]

    On peut réduire le nombre de dimensions en appliquant PCA (dans ce cas, on fait du \textit{principal components regression} ou PCR\index{PCR}). Ceci suppose que les directions dans lesquels \(X_1, \dots, X_p\) ont la plus grande variation ont une relation avec \(Y\).

    Une autre solution est d'utiliser le \textit{best subset selection}\index{Best subset selection} :
    \begin{align*}
        &\min_{\beta} \left\{\sum_{i=1}^n \left(y_i - \beta_0 - \sum_{j=1}^p \beta_j x_{ij}\right)^2\right\}\\
        &\text{sous contrainte } \sum_{j=1}^p \identity(\beta_j \not= 0) \leq s
    \end{align*}
    avec \(s \geq 0\) un hyper-paramètre. Il faut considérer \(p \choose s\) modèles contenant \(s\) prédicteurs. Ceci est infaisable quand \(p\) et \(s\) sont grands. De plus, quand l'espace de recherche est grand, il y a de grands risques d'overfitting. On peut utiliser les stepwise procedures pour contrecarrer ça (mais ces méthodes ne garantissent pas d'obtenir l'optimal).

    \subsection{Shrinkage}
        Finalement, la dernière solution du cours est le \textit{shrinkage}\index{Shrinkage}.
        \begin{definition}
            Le \textit{shrinkage} (ou \textit{regularization}\index{Regularization|see {Shrinkage}}) consiste à fitter un modèle utilisant les \(p\) variables mais les coefficients estimés sont réduits vers zéro par rapport aux coefficients de OLS. Ceci a pour effet de réduire la variance et de faire de la sélection de variables.
        \end{definition}

        \subsubsection{Ridge regression}
            \begin{definition}
                La \textit{régression Ridge}\index{Ridge} est le problème d'optimisation suivant :
                \begin{align*}
                    &\min_{\beta} \left\{\sum_{i=1}^n \left(y_i - \beta_0 - \sum_{j=1}^p \beta_j x_{ij}\right)^2\right\}\\
                    &\text{sous contrainte } \sum_{j=1}^p \beta_j^2 \leq s
                \end{align*}
                avec \(s \leq 0\) un hyper-paramètre.

                Si \(s = 0\), \(\betaRidge = (0, \dots, 0)\). Si \(s = +\infty, \betaRidge = \betaLS\). Si \(s \in ]0, +\infty[\), on fait du tradeoff.
            \end{definition}

            Géométriquement, la contrainte de régularisation est une sphère.

            \begin{definition}
                On peut formuler la régression ridge comme :
                \[
                    \min_{\beta} \sum_{i=1}^n \left(y_i - \beta_0 - \sum_{j=1}^p \beta_j x_{ij}\right)^2 + \lambda \sum_{j=1}^p \beta_j^2
                \]
                avec \(\lambda \leq 0\) un hyper-paramètre.

                Si \(\lambda = 0\), \(\betaRidge = \betaLS\). Si \(\lambda = \infty\), \(\betaRidge = (0, \dots, 0)\). Si \(\lambda \in (0, \infty)\), on fait du tradeoff.

                On peut résoudre ce problème par augmentation des données :
                \[
                    \betaRidge = (X^T X + \lambda \identity_p)^{-1} X^T y
                \]
            \end{definition}

            \paragraph{Un cas simple}
                Supposons que \(n = p\) et que \(X = \identity_n = \identity_p\), alors :
                \begin{align*}
                    &\min_\beta \sum_{j=1}^p (y_j - \beta_j)^2 &\implies \betaLS &= y_j\\
                    &\min_\beta \sum_{j=1}^p (y_j - \beta_j)^2 + \lambda\sum_{j=1}^p \beta_j^2 &\implies \betaRidge &= \frac{y_j}{1+\lambda} = \frac{\betaLS}{1 + \lambda}
                \end{align*}

            \paragraph{Scaling}
                Les coefficients OLS sont `scale equivariant' (multiplier \(X_j\) par une constante \(c \implies\) les coefficients OLS sont divisés par \(c\); peu importe comment la \(j\)\ieme{} variable est scalée, \(X_j\betaEstimated{j}\) est toujours le même). En revanche, les coefficients Ridge peuvent changer drastiquement (à cause de la somme des coefficients au carré).

            \paragraph{Biais et variance}
                Posons \(R = X^T X\)
                \begin{align*}
                    \betaRidge &= (X^T X + \lambda \identity_p)^{-1} X^T y = (R + \lambda \identity_p)^{-1} R(R^{-1} X^T y)\\
                    &= (R + \lambda \identity_p)^{-1} R \betaLS = \left[R(\identity_p + \lambda R^{-1})\right]^1 R \betaLS\\
                    &= (\identity_p + \lambda R^{-1})^{-1} \betaLS
                \end{align*}
                
                Posons \(W_\lambda = (\identity_p + \lambda R^{-1})^{-1}\)
                \begin{align*}
                    \expectation[\betaRidge] &= \expectation[W_\lambda \betaLS] = W_\lambda B \not= \beta & \text{Si \(\lambda \not= 0\)}\\
                    \variance[\betaRidge] &= \variance[W_\lambda \betaLS] = W_\lambda \variance[\betaLS] W_\lambda^T \leq \variance[\betaLS]
                \end{align*}
                On a donc un biais mais un plus petite variance.

            \paragraph{Décomposition en valeurs singulières}
                On peut utiliser la décomposition en valeurs singulières\index{SVD}, c'est-à-dire \(X = U D V^T\). Donc,
                \begin{align*}
                    X^T X &= (UDV^T)^T UDV^T\\
                    &= VD^TU^T UDV^T\\
                    &= VD^T DV^T & \text{Car les colonnes de \(U\) sont orthonormales}\\
                    &= V D^2 V^T & \text{Car \(D\) est diagonale}\\
                    \\
                    \betaLS &= (X^T X)^{-1} X^T y & \text{Définition}\\
                    &= V D^{-1} U^T y & X = U D V^T \text{ et \(V^{-1} = V^T\) (car colonnes orthonormales)}\\
                    &= V D^{-2} D U^T y\\
                    \betaRidge &= (X^T X + \lambda \identity_p)^{-1} X^T y & \text{Définition}\\
                    &= (V D^2 V^T + \lambda VV^T)^{-1} VDU^T y & VV^T = \identity_p \text{ car colonnes orthonormales}\\
                    &= (V(D^2 + \lambda \identity_p) V^T)^{-1} V D U^T y\\
                    &= V(D^2 + \lambda \identity_p)^{-1} V^T V D U^T y\\
                    &= V(D^2 + \lambda \identity_p)^{-1} D U^T y
                \end{align*}
                Les deux expressions ont quasiment la même formulation à une différence près : \(+ \lambda\) pour Ridge. On peut aussi déterminer l'expression des \(\estimation{y}\) :

                \begin{align*}
                    \yLS &= X \betaLS\\
                    &= U D V^T V D^{-2} D U^T y\\
                    &= U D D^{-2} D U^T y\\
                    &= U U^T y\\
                    \yRidge &= X\betaRidge\\
                    &= X V (D^2 + \lambda \identity_p)^{-1} D U^T y\\
                    &= U D V^T V (D^2 + \lambda \identity_p)^{-1} D U^T y\\
                    &= U D (D^2 + \lambda \identity_p)^{-1} D U^T y\\
                    &= U \diagonal(\frac{d_j^2}{d_j^2 + \lambda}) U^T y & \text{\(D\) est diagonale}
                \end{align*}

                Comme \(\lambda \geq 0\), on a \(\frac{d_j^2}{d_j^2 + \lambda} \leq 1\). On a donc bien un effet de shrinkage. Quand \(d_j^2\) est petit, le shrinkage est plus important. La variable \(z_j = X v_j = u_j d_j\) est la \(j\)\ieme{} PC de \(X\).

        \subsubsection{Lasso}
            \begin{definition}
                La \textit{régression Lasso}\index{Lasso}\nomenclature{Lasso}{Least absolute shrinkage and selection operator} est le problème d'optimisation :
                \begin{align*}
                    &\min_\beta \left\{\sum_{i=1}^n \left(y_i - \beta_0 - \sum_{j=1}^p \beta_j x_{ij}\right)^2\right\}\\
                    &\text{sous contrainte} \sum_{j=1}^p |\beta_j| \leq s
                \end{align*}
                avec \(s \geq 0\) un hyper-paramètre.

                Notons \(\betaLasso\) l'estimation des coefficients pour la régression Lasso.

                Si \(s = 0, \betaLasso = (0, \dots, 0)\). Si \(s = \infty, \betaLasso = \betaLS\). Si \(s \in ]0, \infty[\), on fait du tradeoff.
            \end{definition}

            \begin{definition}
                De manière équivalente, on peut exprimer la régression Lasso comme :
                \[
                    \min_\beta \sum_{i=1}^n \left(y_i - \beta_0 - \sum_{j=1}^p \beta_j x_{ij}\right)^2 + \lambda \sum_{j=1}^p |\beta_j|
                \]
                où \(\lambda \geq 0\) est un hyper-paramètre.

                Si \(s = 0, \betaLasso = \betaLasso\). Si \(s = \infty, \betaLasso = (0, \dots, 0)\). Si \(s \in ]0, \infty[\), on fait du tradeoff.
            \end{definition}

            \paragraph{Un cas simple}
                Supposons que \(n = p\) et que \(X = \identity_p = \identity_n\). Dans ce cas, la régression Lasso est 
                \[
                     \min_\beta \sum_{j=1}^p (y_j - \beta_j)^2 + \lambda \sum_{j=1}^p |\beta_j|
                \]
                Alors, on a :
                \[
                    \betaLasso_j = \begin{cases}
                        y_j - \frac{\lambda}{2} & \text{si } y_j > \frac{\lambda}{2}\\
                        y_j + \frac{\lambda}{2} & \text{si } y_j < \frac{-\lambda}{2}\\
                        0                       & \text{si } |y_j| \leq \frac{\lambda}{2}
                    \end{cases}
                \]

                La régression Lasso shrink les coefficients OLS vers zéro par la même constante \(\frac{\lambda}{2}\) et entièrement vers zéro quand ces coefficients sont plus petits que \(\frac{\lambda}{2}\) en valeur absolue.
    \subsection{Classification}
    \begin{lstlisting}
        # Régression logistique
        glm(formule, data = data, family = binomial)
        # LDA
        lda(formule, data = data)
        # QDA
        qda(formule, data = data)
        # k-NN (attention, il faut avoir train.X, train.y et test.X)
        knn(train.X, test.X, train.y, k = 1)
        # Arbres
        tree(formule, data = data)
        # Forêts aléatoires en utilisant m variables et n arbres
        randomForest(formule, data = data, mtry = m, ntrees = n)
    \end{lstlisting}

    \subsubsection{Récupérer probabilités}
        \lstinline{predict} supporte le paramètre \lstinline{type = "prob"} pour récupérer les probabilités pour certains modèles. Pour d'autres (comme \lstinline{glm}), il faut \lstinline{type = "response"}.
    \section{Sélection de modèle}
    \begin{definition}
        La \textit{procédure de sélection de modèle} :
        \begin{enumerate}
            \item Génération des modèles : on génère un ensemble des modèles à essayer.
            \item Validation des modèles : on évalue la performance des modèles en calculant l'erreur de validation, c'est-à-dire une estimation de l'erreur de test.
            \item Sélection d'un modèle : on choisit un modèle. Typiquement, on prend celui qui minimise l'erreur de validation.
        \end{enumerate}
    \end{definition}

    \subsection{Erreurs de training vs de test pour la régression}
        Prenons la régression linéaire. Les paramètres sont estimés par \(\betaEstimated{} = (X^TX)^{-1}X^Ty\), c'est-à-dire la solution OLS\index{OLS}. Les réponses de la régression sont données par :
        \[\estimation{y} = X\betaEstimated{} = X(X^TX)^{-1} X^T y = Hy\]
        avec \(H = X(X^T X)^{-1} X^T\) la \textit{matrice hat}\index{Hat matrix}.

        Pour la régression, \(\betaEstimated{}\) est calculé en minimisant le training MSE\index{MSE}. Notons aussi \(\widetilde{\beta}\) l'estimation obtenue avec le test MSE.
        
        \begin{theorem}[Espérances et MSE]
            On a que :
                \[
                    \expectation\left[\frac{1}{n}\sum_{i=1}^n \left(y_i - \betaEstimated{}^T x_i\right)^2\right] \leq \expectation\left[\frac{1}{m} \sum_{i=1}^m \left(\widetilde{y_i} - \betaEstimated{}^T \widetilde{x_i}\right)^2\right]
                \]

            En d'autres termes, l'espérance du training MSE est inférieure ou égale à l'espérance du test MSE.
        \end{theorem}
        \begin{proof}
            On a que \(\expectation\left[\frac{1}{m} \sum_{i=1}^m \left(\widetilde{y_i} - \betaEstimated{}^T \widetilde{x_i}\right)^2\right] = \expectation\left[\frac{1}{n} \sum_{i=1}^n \left(\widetilde{y_i} - \betaEstimated{}^T \widetilde{x_i}\right)^2\right]\) car le nombre de points ne change pas l'espérance.

            Par facilité, notons \(A = \frac{1}{n}\sum_{i=1}^n \left(y_i - \betaEstimated{}^T\right)^2\) et \(B = \frac{1}{n} \sum_{i=1}^n \left(\widetilde{y_i} - \widetilde{\beta}^T \widetilde{x_i}\right)^2\) (attention au fait que le \(\beta\) de \(B\) a un tilde et non un chapeau !).

            \(A\) et \(B\) ont la même distribution (vu que \(\{(y_i, x_i)\}_{i=1}^n\) et \(\{(\widetilde{y_i}, \widetilde{x_i})\}_{i=1}^n\) suivent la même distribution). Donc, \(\expectation[A] = \expectation[B]\).

            \(B = \frac{1}{n}\sum_{i=1}^n \left(\widetilde{y_i} - \widetilde{\beta}^T \widetilde{x_i}\right)^2 \leq \frac{1}{n} \sum_{i=1}^n \left(\widetilde{y_i} - \betaEstimated{}^T \widetilde{x_i}\right)^2\) par optimalité de l'estimation OLS.

            Donc, \(\expectation[B] \leq \expectation\left[\frac{1}{n} \sum_{i=1}^n (\widetilde{y_i} - \betaEstimated{}^T \widetilde{x_i})^2\right]\).

            Finalement, on a \(\expectation[A] = \expectation[B]\) et \(\expectation[B] \leq \expectation\left[\frac{1}{n} \sum_{i=1}^n (\widetilde{y_i} - \betaEstimated{}^T \widetilde{x_i})^2\right]\). Donc :
            \[
                \expectation\left[\frac{1}{n} \sum_{i=1}^n (y_i - \betaEstimated{}^T x_i)^2\right] = \expectation\left[\frac{1}{m} \sum_{i=1}^m (\widetilde{y_i} - \betaEstimated{}^T \widetilde{x_i})^2\right]
            \]
        \end{proof}

        On veut avoir la plus petite erreur de test pour un nouveau point aléatoire \((x_0, y_0)\) où \(x_0\) et \(y_0\) sont aléatoires. On peut simplifier ce problème. On va garder les mêmes \(x\) mais générer de nouveaux \(y\) bruités. Donc, on fit le modèle sur \(y_i = x_i^T\beta + \varepsilon_i\) mais on prédit sur des échantillons générés comme \(y_i' = x_i^T\beta + \varepsilon_i'\) où \(\varepsilon_i\) et \(\varepsilon_i'\) sont iid. On va tester si le modèle entraîné sur \((x_i, y_i)\) donne des bonnes prédictions pour \((x_i, y_i')\).

        Soit \(\estimation{m_i} = x_i \betaEstimated{}\). On veut comparer \(\expectation\left[\frac{1}{n} \sum_{i=1}^n (y_i' - \estimation{m_i})^2\right]\) avec \(\expectation\left[\frac{1}{n} \sum_{i=1}^n (y_i - \estimation{m_i})^2\right]\). Remarquons que \(y_i\) et \(\estimation{m_i}\) sont des variables aléatoires dépendantes (pas indépendantes) puisque \(\estimation{m_i}\) dépend de \(y_i\) (à travers \(\betaEstimated{}\)). En revanche, \(y_i'\) et \(\estimation{m_i}\) sont indépendants.

        On a :
        \begin{align*}
            \expectation\left[(y_i - \estimation{m_i})^2\right] &= \variance[y_i - \estimation{m_i}] + (\expectation[y_i - \estimation{m_i}])^2 & \text{Car } \variance^2[X] = \expectation[X^2] - \expectation^2[X]\\
            &= \variance[y_i] + \variance[\estimation{m_i}] - 2\covariance[y_i, \estimation{m_i}] + (\expectation[y_i] - \expectation[\estimation{m_i}])^2 & \text{Propriétés variance et espérance}\\
            \expectation\left[(y_i' - \estimation{m_i})^2\right] &= \variance[y_i' - \estimation{m_i}] + (\expectation[y_i' - \estimation{m_i}])^2\\
            &= \variance[y_i'] + \variance[\estimation{m_i}] - 2\covariance[y_i', \estimation{m_i}] + (\expectation[y_i'] - \expectation[\estimation{m_i}])^2
        \end{align*}

        \(y_i\) et \(y_i'\) sont indépendants mais ont la même distribution. Donc, \(\expectation[y_i] = \expectation[y_i']\) et \(\variance[y_i] = \variance[y_i']\). On a aussi que \(\covariance[y_i', \estimation{m_i}] = 0\) (car variables indépendantes).

        Donc :
        \begin{align*}
            \expectation\left[(y_i' - \estimation{m_i})^2\right] &= \variance[y_i] + \variance[\estimation{m_i}] + (\expectation[y_i] - \expectation[\estimation{m_i}])^2 = \expectation\left[(y_i - \estimation{m_i})^2\right] + 2\covariance[y_i, \estimation{m_i}]\\
            \expectation\left[\frac{1}{n} \sum_{i=1}^n (y_i' - \estimation{m_i})^2\right] &= \expectation\left[\frac{1}{n} \sum_{i=1}^n (y_i - \estimation{m_i})^2\right] + \frac{2}{n} \sum_{i=1}^n \covariance[y_i, \estimation{m_i}]\\
            &\approx \frac{1}{n} \sum_{i=1}^n (y_i - \estimation{m_i})^2 + \frac{2}{n} \sum_{i=1}^n \covariance[y_i, \estimation{m_i}]
        \end{align*}

        \begin{definition}
            Le training error sous-estime systématiquement le test error. L'\textit{optimisme}\index{Optimisme} est la mesure de ce qui est sous-estimé.

            Une façon d'estimer le test error est d'estimer l'optimisme et de l'ajouter au training error (voir AIC et BIC)\index{AIC}\index{BIC}.

            Une autre façon est de directement estimer le test error via des méthodes de resampling\index{Resampling} (voir validation set, cross-validation et bootstrap).
        \end{definition}

        \begin{exemple}
            Pour la régression linéaire, supposons que \(X \in \R^{n \times (p+1)}\) (en rajoutant une colonne pour \(\beta_0\) dans la matrice). On suppose aussi que \(X\) et \(\beta\) ne sont pas aléatoires. Par conséquent, \(H\) n'est pas aléatoire non plus. Commençons par montrer que \(\covariance[y_i, \estimation{m_i}] = \sigma^2 H_{ii}\). Pour ça, montrons que \(\covariance[y, \estimation{m}] = \sigma^2 H\) (on passe à la forme matricielle; l'espérance d'un vecteur est le vecteur des espérances) :
            \begin{align*}
                \covariance[y, \estimation{m}] &= \covariance[y, X\betaEstimated{}] & \text{Définition de \(m\)}\\
                &= \covariance[y, Hy] & \text{Définition de \(H\)}\\
                &= \expectation[(y - \expectation[y])(Hy - \expectation[Hy])] & \text{Définition de \(\covariance\)}\\
                &= \expectation[(f(X) + \varepsilon - \expectation[f(X) + \varepsilon]) (Hy - \expectation[Hy])] & \text{Régression linéaire}\\
                &= \expectation[(\varepsilon)(H (y - \expectation[y]))] & \text{Linéarité de l'espérance et \(X\) pas aléatoire}\\
                &= \expectation[\varepsilon^2 H] = H \expectation[\varepsilon^2] = H \sigma^2 & \expectation[\varepsilon^2] = \expectation[\varepsilon]^2 + \variance[\varepsilon] = \variance[\varepsilon]
            \end{align*}

            Donc, \(\covariance[y_i, \estimation{m_i}] = \sigma^2 H_{ii}\) en prenant l'élément \(i, i\) dans les deux matrices. A partir de là, on a :

            \begin{align*}
                \frac{2}{n} \sum_{i=1}^n \covariance[y_i, \estimation{m_i}] &= \frac{2}{n} \sum_{i=1}^n \sigma^2 H_{ii} & \text{Voir au-dessus}\\
                &= \frac{2\sigma^2}{n} \sum_{i=1}^n H_{ii}\\
                &= \frac{2}{n} \sigma^2 \trace(H) & \text{Définition trace}\\
                &= \frac{2}{n} \sigma^2 \trace(X (X^T X)^{-1} X^T) & \text{Definition \(H\)}\\
                &= \frac{2}{n} \sigma^2 \trace(X^T X (X^T X)^{-1}) & \trace(AB) = \trace(BA)\\
                &= \frac{2}{n} \sigma^2 \trace(I_{p+1}) & X \in \R^{n \times (p + 1)}\\
                &= \frac{2}{n} \sigma^2 (p + 1)
            \end{align*}
            et :
            \[
                \expectation\left[\frac{1}{n} \sum_{i=1}^n (y_i' - \estimation{m_i})^2\right] \approx \frac{1}{n} \sum_{i=1}^n (y_i - \estimation{m_i})^2 + \frac{2}{n} \sigma^2 (p + 1)
            \]

            L'optimisme est donc \(\frac{2}{n} \sigma^2 (p + 1)\), croît avec \(\sigma^2\) et \(p\) et décroît avec \(n\).
        \end{exemple}

    \subsection{Mesures et méthodes}
        Minimiser le RSS\index{RSS} va toujours choisir le modèle avec le plus de variables possibles.

        \begin{definition}
            L'\textit{Estimated Residual Variance}\index{Estimated Residual Variance} :
            \[
                \estimation{\sigma}^2 = \frac{\RSS}{n - p -1}
            \]
            avec \(p\) le nombre de variables.
        \end{definition}

        Minimiser \(\estimation{\sigma}^2\) marche plutôt bien pour choisir les variables (mais on privilégie de meilleurs méthodes; voir plus loin). Pour rappel, \(\estimation{\sigma}^2 = \MSE \frac{1}{1 - (p + 1)/n}\). Par le développement de Taylor, on a que \((1 - x)^{-1} = 1 + x + x^2 + \dots\). On tronque la série à \(1 + x\). Pour un \(p\) fixé, l'approximation devient exacte quand \(n \to \infty\) :
        \[
            \estimation{\sigma}^2 \approx \MSE (1 + \frac{p + 1}{n}) = \MSE + \MSE\frac{p + 1}{n}
        \]

        Même dans les cas où \(\MSE\) est un estimateur `consistent' de \(\sigma^2\), la pénalité est la moitié de ce qu'elle devrait être\footnote{Je ne sais pas d'où ça sort} : \(\MSE + 2\sigma^2 \frac{p+1}{n}\).

        La R-squared statistic\index{R-squared} donne la proportion de la variance expliquée et est indépendante de l'échelle de \(y\). Cependant, R-squared ne permet de `degré de liberté' et ajouter une variable (n'importe laquelle) a tendance à augmenter R-squared (même si cette variable est inutile). Pour résoudre ça, on utilise l'adjusted R-squared.

        \begin{definition}
            L'\textit{adjusted R-squared}\index{Adjusted R-squared} :
            \[
                \adjustedRSquared = 1 - (1 - R^2) \frac{n-1}{n - p - 1}
            \]
        \end{definition}

        Maximiser \(\adjustedRSquared\) est équivalent à minimiser \(\estimation{\sigma}^2\). \(\adjustedRSquared\) est meilleur de que \(R^2\) mais ne va pas marcher très bien.

        \begin{definition}[\(\AIC\) et \(AIC_C\)]
            Le \textit{Akaike's Information Criterion}\index{Akaike's Information Criterion|see {AIC}} (noté \textit{AIC}) :
            \[
                \AIC = -2 \log(L) + 2(p+1)
            \]
            avec \(L\) la likelihood et \(p\) le nombre de variables dans le modèle.

            Il s'agit d'une approche \textit{penalized likelihood}. Minimiser le AIC donne le meilleur modèle pour la prédiction.

            AIC pénalise plus que \(\adjustedRSquared\).

            Minimiser AIC est asymptotiquement équivalent à minimiser MSE via du leave-one-out cross-validation.

            Pour des petites valeurs de \(n\), l'AIC a tendance à sélectionner trop de variables. Pour contrer ça, une version qui corrige le biais de l'AIC a été développée\index{Corrected AIC} (notée \(\AIC_C\)\index{AICC@\(\AIC_C\)}) :
            \[
                \AIC_C = \AIC + \frac{2(p + 2)(p + 3)}{n - p - 1}
            \]
            Comme pour l'\(\AIC\), le \(\AIC_C\) doit être minimisé.
        \end{definition}

        \begin{definition}
            Le \textit{Schwartz Bayesian Information Criterion}\index{Schwartz Bayesian Information Criterion|see {BIC}} (noté \textit{BIC}\index{BIC} ou \textit{SBIC}\index{SBIC|see {BIC}} ou \textit{SC}\index{SC|see {BIC}}) :
            \[
                \BIC = -2\log(L) + (p + 1)\log(n)
            \]
            avec \(L\) la likelihood et \(p\) le nombre de variables dans le modèle.

            \(\BIC\) pénalise plus que \(\AIC\).

            Minimiser \(\BIC\) est asymptotiquement équivalent à faire du leave-\(v\)-out cross-validation quand \[v = n\left(1 - \frac{1}{\log(n) - 1}\right)\]
        \end{definition}

        Plusieurs méthodes pour trouver les meilleures variables pour une régression :
        \begin{itemize}
            \item \textit{Best subsets regression}\index{Best subsets regression} : fit tous les modèles de régression avec au moins une variable et choisir le meilleur modèle (basé sur CV, \(\AIC\) ou \(\AIC_C\)). Impossible s'il y a beaucoup de variables.
            \item \textit{Backwards stepwise}\index{Backwards stepwise} : commencer avec un modèle contenant toutes les variables, en retirer une et garder le modèle qui a la plus petite valeur (pour CV, \(\AIC\) ou \(\AIC_C\)). Recommencer jusqu'à satisfaction.
            \item \textit{Forward stepwise} : commencer avec aucune variable, ajouter celle qui donne le meilleur CV, \(\AIC\) ou \(\AIC_C\).
        \end{itemize}
        \begin{remarque}
            Les stepwise ne garantissent pas de trouver le meilleur modèle possible.
        \end{remarque}
    \section{Resampling}
    \begin{definition}
        Les méthodes de \textit{resampling}\index{Resampling} sont utilisées dans :
        \begin{enumerate}
            \item La validation de modèle en utilisant des sous-ensembles des données (cross-validation et bootstrapping);
            \item L'estimation de l'incertitude des statistiques en tirant aléatoirement (avec remise) des données (bootstrapping):
            \item Faire des tests pour voir si une méthode est significative (tests de permutation)
        \end{enumerate}
    \end{definition}

    \subsection{\(k\)-fold cross-validation}
        \begin{definition}
            La \textit{\(k\)-fold cross-validation}\index{Cross-validation}
            \begin{itemize}
                \item Diviser l'ensemble des données en \(k\) parties différentes
                \item Retirer une partie, entraîner le modèle sur les \(k - 1\) autres parties et calculer le \(\MSE\) sur la partie retirée
                \item Répéter \(k\) fois (une fois par partie à retirer)
            \end{itemize}

            En faisant la moyenne des \(k\) MSE, on obtient une estimation de l'erreur de validation (de test) pour de nouvelles observations :
            \[
                \CV_{k} = \frac{1}{k} \sum_{i=1}^k \MSE_i
            \]

            Le \textit{leave-one-out cross-validation}\index{Leave-one-out} est un cas spécial où \(k = n\).
        \end{definition}

        Chaque ensemble d'entraînement est \(\frac{k - 1}{k}\) fois aussi grand que l'ensemble de base. Donc, les estimations de l'erreur de prédiction sont biaisées vers le haut. Le biais est minimisé quand \(k = n\). Cependant, la variance augmente avec \(k\) (car il y a des observations communes entre les différents ensembles). On prend typiquement \(k = 5\) ou \(k = 10\) pour un bon compromis entre le biais et la variance.

        Il faut faire attention à bien utiliser la cross-validation. Par exemple, :
        \begin{enumerate}
            \item Diviser l'ensemble en 10 folds
            \item Pour \(i = 1, \dots, 10\), on utilise tous les folds sauf \(i\), on trouver les 5 meilleures variables, on entraîne un modèle avec ces 5 variables et on calcule l'erreur sur le fold \(i\).
            \item On calcule la moyenne des erreurs.
        \end{enumerate}
        En bref, \textbf{toutes les étapes} de la procédure doivent être dans la cross-validation.

        On choisit le modèle le plus simple dont l'erreur de CV n'est pas plus grande qu'un écart-type au-dessus que l'erreur du modèle avec la plus petite erreur.

    \subsection{Bootstrap}
        \begin{definition}
            Le \textit{bootstrap}\index{Bootstrap} est un outil statistique permettant de quantifier l'incertitude associée à un estimateur ou une méthode d'apprentissage.

            On imite le processus d'obtention de nouveaux jeux de données pour estimer la variabilité de notre estimation. Pour ça, on peut tirer des observations depuis le jeu de données original avec remise (nonparamétrique) ou depuis un modèle estimé (paramétrique). On obtient des \textit{échantillons bootstrap}.

            L'\cref{alg:bootstrap} donne un pseudo-code pour le bootstrap.
        \end{definition}

        \begin{algorithm}
            \caption{Bootstrap}
            \label{alg:bootstrap}
            \begin{algorithmic}[1]
                \State Trouver une bonne estimation \(\estimation{P}\) de \(P\) (la distribution originale)
                \State Tirer \(B\) échantillons bootstrap indépendants \(Z^{*(1)}, \dots, Z^{*(B)}\) de \(\estimation{P}\) : \(\forall b \in \{1, \dots, B\}, Z_1^{*(b)}, \dots, Z_n^{*(b)} \sim \estimation{P}\)
                \State Evaluer : \(\forall b \in \{1, \dots, B\}, \estimation{\theta}^{*(b)} = s(Z^{*(b)})\)
                \State Estimater la mesure qui nous intéresse depuis la distribution des \(\estimation{\theta}^{*(b)}\)
            \end{algorithmic}
        \end{algorithm}

        Bootstrap peut servir pour estimer beaucoup de choses (contrairement à la cross-validation qui ne gère que les erreurs). Par exemple, on peut chercher \(\estimation{\theta} = \) moyenne, médianne, etc.

        Le nombre \(B\) d'échantillons et la taille \(n\) des échantillons (le nombre d'observations par échantillon) jouent tous deux un rôle dans la précision de l'estimation.

        \subsubsection{Estimation de l'erreur de prédiction}
            On fit le modèle sur des échantillons bootstrap (avec le nombre d'observations par échantillon que dans le jeu de base) et on regarde la qualité des prédictions :
            \[
                \errorBoot = \frac{1}{B}\frac{1}{N} \sum_{b=1}^B \sum_{i=1}^N L(y_i, \estimation{f}^{*b}(x_i))
            \]

            Comme on tire avec remise et que chaque échantillon a la même taille que le jeu de base, il y a des observations qui peut peuvent apparaître plus qu'une fois dans les échantillons et d'autres qui n'apparaissent jamais. Donc, on a que nos ensembles d'entraînement et de validation ont des observations en commun (ce qui nous donne une estimation erronée de l'erreur de validation). Nous allons calculer la probabilité qu'un observation \(i\) donnée soit dans l'échantillon bootstrap \(b\) :
            \begin{align*}
                P(\text{observation \(i\) soit tirée}) &= \frac{1}{n} & \text{Distribution uniforme}\\
                P(\text{observation \(i\) ne soit pas tirée}) &= 1 - \frac{1}{n}\\
                P(\text{observation } i \not\in \text{échantillon } b) &= (1 - \frac{1}{n})^n & \text{Il faut que l'observation ne soit jamais tirée}\\
                P(\text{observation \(i \in\) échantillon \(b\)}) &= 1 - (1 - \frac{1}{n})^n \\
                &\approx 1 - \frac{1}{e} \approx 0.632
            \end{align*}

            On a donc que chaque observation a 63\% de chances d'apparaître dans un échantillon bootstrap. Nous allons améliorer bootstrap en mémorisant les observations qui ne sont pas dans les échantillons. Ceci permet de mieux estimer l'erreur de validation :
            \[
                \errorLooBoot = \frac{1}{N} \sum_{i=1}^N \frac{1}{|C^{-i}|} \sum_{b \in C^{-i}} L(y_i, \estimation{f}^{*b} (x_i))
            \]
            avec \(C^{-i}\) l'ensemble des indices des échantillons bootstrap qui ne contiennent pas l'observation \(i\).

            Comme pour la cross-validation, il y a un biais `training-set-size'.
    \subsection{Réduction de dimensionnalité (PCA)}
    \begin{lstlisting}
        # PCA en scalant et centrant les données
        PCA = prcomp(data, scale = TRUE, center = TRUE)
        # Appliquer la transformation sur de nouvelles données
        predict(PCA, newdata)
    \end{lstlisting}
    \section{Régression avancée}
    Rappel : solution optimale de la régression linéaire donnée par OLS : \(\betaLS = (X^TX)^{-1} X^Ty\)

    Cette solution peut avoir des problèmes en haute dimension parce qu'il peut arriver que certaines colonnes de \(X\) ne sont pas linéairement indépendantes. Par conséquent, \(X\) n'est pas `full rank'. Dans ce cas, \(X^T X\) est singulière (non-inversible) et les coefficients OLS n'ont pas une unique valeur.

    Pour rappel, le biais de \(\betaLS\) est \(\expectation[\betaLS] - \beta^* = 0\) et la variance est \(\variance[\betaLS] = \sigma^2 (X^T X)^{-1}\). Si \(X\) a des colonnes orthonormales :
    \[
        \trace(\variance[\betaLS]) = \sigma^2 p
    \]

    On peut réduire le nombre de dimensions en appliquant PCA (dans ce cas, on fait du \textit{principal components regression} ou PCR\index{PCR}). Ceci suppose que les directions dans lesquels \(X_1, \dots, X_p\) ont la plus grande variation ont une relation avec \(Y\).

    Une autre solution est d'utiliser le \textit{best subset selection}\index{Best subset selection} :
    \begin{align*}
        &\min_{\beta} \left\{\sum_{i=1}^n \left(y_i - \beta_0 - \sum_{j=1}^p \beta_j x_{ij}\right)^2\right\}\\
        &\text{sous contrainte } \sum_{j=1}^p \identity(\beta_j \not= 0) \leq s
    \end{align*}
    avec \(s \geq 0\) un hyper-paramètre. Il faut considérer \(p \choose s\) modèles contenant \(s\) prédicteurs. Ceci est infaisable quand \(p\) et \(s\) sont grands. De plus, quand l'espace de recherche est grand, il y a de grands risques d'overfitting. On peut utiliser les stepwise procedures pour contrecarrer ça (mais ces méthodes ne garantissent pas d'obtenir l'optimal).

    \subsection{Shrinkage}
        Finalement, la dernière solution du cours est le \textit{shrinkage}\index{Shrinkage}.
        \begin{definition}
            Le \textit{shrinkage} (ou \textit{regularization}\index{Regularization|see {Shrinkage}}) consiste à fitter un modèle utilisant les \(p\) variables mais les coefficients estimés sont réduits vers zéro par rapport aux coefficients de OLS. Ceci a pour effet de réduire la variance et de faire de la sélection de variables.
        \end{definition}

        \subsubsection{Ridge regression}
            \begin{definition}
                La \textit{régression Ridge}\index{Ridge} est le problème d'optimisation suivant :
                \begin{align*}
                    &\min_{\beta} \left\{\sum_{i=1}^n \left(y_i - \beta_0 - \sum_{j=1}^p \beta_j x_{ij}\right)^2\right\}\\
                    &\text{sous contrainte } \sum_{j=1}^p \beta_j^2 \leq s
                \end{align*}
                avec \(s \leq 0\) un hyper-paramètre.

                Si \(s = 0\), \(\betaRidge = (0, \dots, 0)\). Si \(s = +\infty, \betaRidge = \betaLS\). Si \(s \in ]0, +\infty[\), on fait du tradeoff.
            \end{definition}

            Géométriquement, la contrainte de régularisation est une sphère.

            \begin{definition}
                On peut formuler la régression ridge comme :
                \[
                    \min_{\beta} \sum_{i=1}^n \left(y_i - \beta_0 - \sum_{j=1}^p \beta_j x_{ij}\right)^2 + \lambda \sum_{j=1}^p \beta_j^2
                \]
                avec \(\lambda \leq 0\) un hyper-paramètre.

                Si \(\lambda = 0\), \(\betaRidge = \betaLS\). Si \(\lambda = \infty\), \(\betaRidge = (0, \dots, 0)\). Si \(\lambda \in (0, \infty)\), on fait du tradeoff.

                On peut résoudre ce problème par augmentation des données :
                \[
                    \betaRidge = (X^T X + \lambda \identity_p)^{-1} X^T y
                \]
            \end{definition}

            \paragraph{Un cas simple}
                Supposons que \(n = p\) et que \(X = \identity_n = \identity_p\), alors :
                \begin{align*}
                    &\min_\beta \sum_{j=1}^p (y_j - \beta_j)^2 &\implies \betaLS &= y_j\\
                    &\min_\beta \sum_{j=1}^p (y_j - \beta_j)^2 + \lambda\sum_{j=1}^p \beta_j^2 &\implies \betaRidge &= \frac{y_j}{1+\lambda} = \frac{\betaLS}{1 + \lambda}
                \end{align*}

            \paragraph{Scaling}
                Les coefficients OLS sont `scale equivariant' (multiplier \(X_j\) par une constante \(c \implies\) les coefficients OLS sont divisés par \(c\); peu importe comment la \(j\)\ieme{} variable est scalée, \(X_j\betaEstimated{j}\) est toujours le même). En revanche, les coefficients Ridge peuvent changer drastiquement (à cause de la somme des coefficients au carré).

            \paragraph{Biais et variance}
                Posons \(R = X^T X\)
                \begin{align*}
                    \betaRidge &= (X^T X + \lambda \identity_p)^{-1} X^T y = (R + \lambda \identity_p)^{-1} R(R^{-1} X^T y)\\
                    &= (R + \lambda \identity_p)^{-1} R \betaLS = \left[R(\identity_p + \lambda R^{-1})\right]^1 R \betaLS\\
                    &= (\identity_p + \lambda R^{-1})^{-1} \betaLS
                \end{align*}
                
                Posons \(W_\lambda = (\identity_p + \lambda R^{-1})^{-1}\)
                \begin{align*}
                    \expectation[\betaRidge] &= \expectation[W_\lambda \betaLS] = W_\lambda B \not= \beta & \text{Si \(\lambda \not= 0\)}\\
                    \variance[\betaRidge] &= \variance[W_\lambda \betaLS] = W_\lambda \variance[\betaLS] W_\lambda^T \leq \variance[\betaLS]
                \end{align*}
                On a donc un biais mais un plus petite variance.

            \paragraph{Décomposition en valeurs singulières}
                On peut utiliser la décomposition en valeurs singulières\index{SVD}, c'est-à-dire \(X = U D V^T\). Donc,
                \begin{align*}
                    X^T X &= (UDV^T)^T UDV^T\\
                    &= VD^TU^T UDV^T\\
                    &= VD^T DV^T & \text{Car les colonnes de \(U\) sont orthonormales}\\
                    &= V D^2 V^T & \text{Car \(D\) est diagonale}\\
                    \\
                    \betaLS &= (X^T X)^{-1} X^T y & \text{Définition}\\
                    &= V D^{-1} U^T y & X = U D V^T \text{ et \(V^{-1} = V^T\) (car colonnes orthonormales)}\\
                    &= V D^{-2} D U^T y\\
                    \betaRidge &= (X^T X + \lambda \identity_p)^{-1} X^T y & \text{Définition}\\
                    &= (V D^2 V^T + \lambda VV^T)^{-1} VDU^T y & VV^T = \identity_p \text{ car colonnes orthonormales}\\
                    &= (V(D^2 + \lambda \identity_p) V^T)^{-1} V D U^T y\\
                    &= V(D^2 + \lambda \identity_p)^{-1} V^T V D U^T y\\
                    &= V(D^2 + \lambda \identity_p)^{-1} D U^T y
                \end{align*}
                Les deux expressions ont quasiment la même formulation à une différence près : \(+ \lambda\) pour Ridge. On peut aussi déterminer l'expression des \(\estimation{y}\) :

                \begin{align*}
                    \yLS &= X \betaLS\\
                    &= U D V^T V D^{-2} D U^T y\\
                    &= U D D^{-2} D U^T y\\
                    &= U U^T y\\
                    \yRidge &= X\betaRidge\\
                    &= X V (D^2 + \lambda \identity_p)^{-1} D U^T y\\
                    &= U D V^T V (D^2 + \lambda \identity_p)^{-1} D U^T y\\
                    &= U D (D^2 + \lambda \identity_p)^{-1} D U^T y\\
                    &= U \diagonal(\frac{d_j^2}{d_j^2 + \lambda}) U^T y & \text{\(D\) est diagonale}
                \end{align*}

                Comme \(\lambda \geq 0\), on a \(\frac{d_j^2}{d_j^2 + \lambda} \leq 1\). On a donc bien un effet de shrinkage. Quand \(d_j^2\) est petit, le shrinkage est plus important. La variable \(z_j = X v_j = u_j d_j\) est la \(j\)\ieme{} PC de \(X\).

        \subsubsection{Lasso}
            \begin{definition}
                La \textit{régression Lasso}\index{Lasso}\nomenclature{Lasso}{Least absolute shrinkage and selection operator} est le problème d'optimisation :
                \begin{align*}
                    &\min_\beta \left\{\sum_{i=1}^n \left(y_i - \beta_0 - \sum_{j=1}^p \beta_j x_{ij}\right)^2\right\}\\
                    &\text{sous contrainte} \sum_{j=1}^p |\beta_j| \leq s
                \end{align*}
                avec \(s \geq 0\) un hyper-paramètre.

                Notons \(\betaLasso\) l'estimation des coefficients pour la régression Lasso.

                Si \(s = 0, \betaLasso = (0, \dots, 0)\). Si \(s = \infty, \betaLasso = \betaLS\). Si \(s \in ]0, \infty[\), on fait du tradeoff.
            \end{definition}

            \begin{definition}
                De manière équivalente, on peut exprimer la régression Lasso comme :
                \[
                    \min_\beta \sum_{i=1}^n \left(y_i - \beta_0 - \sum_{j=1}^p \beta_j x_{ij}\right)^2 + \lambda \sum_{j=1}^p |\beta_j|
                \]
                où \(\lambda \geq 0\) est un hyper-paramètre.

                Si \(s = 0, \betaLasso = \betaLasso\). Si \(s = \infty, \betaLasso = (0, \dots, 0)\). Si \(s \in ]0, \infty[\), on fait du tradeoff.
            \end{definition}

            \paragraph{Un cas simple}
                Supposons que \(n = p\) et que \(X = \identity_p = \identity_n\). Dans ce cas, la régression Lasso est 
                \[
                     \min_\beta \sum_{j=1}^p (y_j - \beta_j)^2 + \lambda \sum_{j=1}^p |\beta_j|
                \]
                Alors, on a :
                \[
                    \betaLasso_j = \begin{cases}
                        y_j - \frac{\lambda}{2} & \text{si } y_j > \frac{\lambda}{2}\\
                        y_j + \frac{\lambda}{2} & \text{si } y_j < \frac{-\lambda}{2}\\
                        0                       & \text{si } |y_j| \leq \frac{\lambda}{2}
                    \end{cases}
                \]

                La régression Lasso shrink les coefficients OLS vers zéro par la même constante \(\frac{\lambda}{2}\) et entièrement vers zéro quand ces coefficients sont plus petits que \(\frac{\lambda}{2}\) en valeur absolue.
    \subsection{Classification}
    \begin{lstlisting}
        # Régression logistique
        glm(formule, data = data, family = binomial)
        # LDA
        lda(formule, data = data)
        # QDA
        qda(formule, data = data)
        # k-NN (attention, il faut avoir train.X, train.y et test.X)
        knn(train.X, test.X, train.y, k = 1)
        # Arbres
        tree(formule, data = data)
        # Forêts aléatoires en utilisant m variables et n arbres
        randomForest(formule, data = data, mtry = m, ntrees = n)
    \end{lstlisting}

    \subsubsection{Récupérer probabilités}
        \lstinline{predict} supporte le paramètre \lstinline{type = "prob"} pour récupérer les probabilités pour certains modèles. Pour d'autres (comme \lstinline{glm}), il faut \lstinline{type = "response"}.
    \section{R}
    Nous donnons ici les fonctions utilisées en TP et pour les devoirs.

    \subsection{Divers}
    \subsubsection{Opérations matricielles}
        Pour multiplier deux matrices \(A\) et \(B\), il faut écrire \lstinline{A %*% B}.

    \subsubsection{Génération aléatoire}
        \begin{lstlisting}
            # Contrôle la graine
            set.seed(1)
            # Générer un vecteur de taille n
            # selon une loi normale de moyenne m et d'écart-type s
            rnorm(n, mean = m, sd = s)
            # selon une loi uniforme entre a et b
            runif(n, min = a, max = b)
            # selon une loi de Poisson avec comme paramètre l
            rpois(n, lambda = l)
        \end{lstlisting}

    \subsubsection{Tidyverse}
        \begin{lstlisting}
            dataset = dataset %>%
                # Ajoute ou transforme une colonne
                mutate(column = transformation(other.column)) %>%
                # Filtre les données (par exemple, filter(!is.na(column)))
                filter(condition(column)) %>%
                # Retire les lignes dupliquées
                distinct(column) %>%
                # Sélectionne des colonnes
                select(col1, col2, ...) %>%
                # Groupe les données en ligne avec la même valeur pour column
                group_by(column) %>%
                # Retire les groupes
                ungroup() %>%
                # Rassemble plusieurs colonnes en une paires (clé, valeurs)
                # Les autres colonnes sont dupliquées
                gather(col1, col2, col3, ..., key = "keycolumn", value = "valuecolumn") %>%
                # Opération inverse (découpe les paires en colonnes)
                spread(key = "keycolumn", value = "valuecolumn")
                # Summarise les données en une seule ligne avec les valeurs
                summarise(new.column = summary.function(column))
        \end{lstlisting}

        Fonctions supportées par \lstinline{summarise} :
        \begin{itemize}
            \item \lstinline{first}, \lstinline{last}, \lstinline{nth} pour récupérer les valeurs
            \item \lstinline{n} pour le nombre de valeurs, \lstinline{n_distinct} pour le nombre de valeurs différentes
            \item \lstinline{min}, \lstinline{max}, \lstinline{mean}, \lstinline{median}, \lstinline{var}, \lstinline{var} pour récupérer les valeurs statistiques associées
        \end{itemize}

    \subsubsection{Couper un jeu de données}
        On va couper aléatoirement un jeu de données en un ensemble d'entraînement (avec 2000 données) et un ensemble de test (avec toutes les autres données).

        \begin{lstlisting}
            train = sample(1:nrow(data), 2000)
            data.train = data[train,]
            data.test = data[-train,]
        \end{lstlisting}
    \section{Régression avancée}
    Rappel : solution optimale de la régression linéaire donnée par OLS : \(\betaLS = (X^TX)^{-1} X^Ty\)

    Cette solution peut avoir des problèmes en haute dimension parce qu'il peut arriver que certaines colonnes de \(X\) ne sont pas linéairement indépendantes. Par conséquent, \(X\) n'est pas `full rank'. Dans ce cas, \(X^T X\) est singulière (non-inversible) et les coefficients OLS n'ont pas une unique valeur.

    Pour rappel, le biais de \(\betaLS\) est \(\expectation[\betaLS] - \beta^* = 0\) et la variance est \(\variance[\betaLS] = \sigma^2 (X^T X)^{-1}\). Si \(X\) a des colonnes orthonormales :
    \[
        \trace(\variance[\betaLS]) = \sigma^2 p
    \]

    On peut réduire le nombre de dimensions en appliquant PCA (dans ce cas, on fait du \textit{principal components regression} ou PCR\index{PCR}). Ceci suppose que les directions dans lesquels \(X_1, \dots, X_p\) ont la plus grande variation ont une relation avec \(Y\).

    Une autre solution est d'utiliser le \textit{best subset selection}\index{Best subset selection} :
    \begin{align*}
        &\min_{\beta} \left\{\sum_{i=1}^n \left(y_i - \beta_0 - \sum_{j=1}^p \beta_j x_{ij}\right)^2\right\}\\
        &\text{sous contrainte } \sum_{j=1}^p \identity(\beta_j \not= 0) \leq s
    \end{align*}
    avec \(s \geq 0\) un hyper-paramètre. Il faut considérer \(p \choose s\) modèles contenant \(s\) prédicteurs. Ceci est infaisable quand \(p\) et \(s\) sont grands. De plus, quand l'espace de recherche est grand, il y a de grands risques d'overfitting. On peut utiliser les stepwise procedures pour contrecarrer ça (mais ces méthodes ne garantissent pas d'obtenir l'optimal).

    \subsection{Shrinkage}
        Finalement, la dernière solution du cours est le \textit{shrinkage}\index{Shrinkage}.
        \begin{definition}
            Le \textit{shrinkage} (ou \textit{regularization}\index{Regularization|see {Shrinkage}}) consiste à fitter un modèle utilisant les \(p\) variables mais les coefficients estimés sont réduits vers zéro par rapport aux coefficients de OLS. Ceci a pour effet de réduire la variance et de faire de la sélection de variables.
        \end{definition}

        \subsubsection{Ridge regression}
            \begin{definition}
                La \textit{régression Ridge}\index{Ridge} est le problème d'optimisation suivant :
                \begin{align*}
                    &\min_{\beta} \left\{\sum_{i=1}^n \left(y_i - \beta_0 - \sum_{j=1}^p \beta_j x_{ij}\right)^2\right\}\\
                    &\text{sous contrainte } \sum_{j=1}^p \beta_j^2 \leq s
                \end{align*}
                avec \(s \leq 0\) un hyper-paramètre.

                Si \(s = 0\), \(\betaRidge = (0, \dots, 0)\). Si \(s = +\infty, \betaRidge = \betaLS\). Si \(s \in ]0, +\infty[\), on fait du tradeoff.
            \end{definition}

            Géométriquement, la contrainte de régularisation est une sphère.

            \begin{definition}
                On peut formuler la régression ridge comme :
                \[
                    \min_{\beta} \sum_{i=1}^n \left(y_i - \beta_0 - \sum_{j=1}^p \beta_j x_{ij}\right)^2 + \lambda \sum_{j=1}^p \beta_j^2
                \]
                avec \(\lambda \leq 0\) un hyper-paramètre.

                Si \(\lambda = 0\), \(\betaRidge = \betaLS\). Si \(\lambda = \infty\), \(\betaRidge = (0, \dots, 0)\). Si \(\lambda \in (0, \infty)\), on fait du tradeoff.

                On peut résoudre ce problème par augmentation des données :
                \[
                    \betaRidge = (X^T X + \lambda \identity_p)^{-1} X^T y
                \]
            \end{definition}

            \paragraph{Un cas simple}
                Supposons que \(n = p\) et que \(X = \identity_n = \identity_p\), alors :
                \begin{align*}
                    &\min_\beta \sum_{j=1}^p (y_j - \beta_j)^2 &\implies \betaLS &= y_j\\
                    &\min_\beta \sum_{j=1}^p (y_j - \beta_j)^2 + \lambda\sum_{j=1}^p \beta_j^2 &\implies \betaRidge &= \frac{y_j}{1+\lambda} = \frac{\betaLS}{1 + \lambda}
                \end{align*}

            \paragraph{Scaling}
                Les coefficients OLS sont `scale equivariant' (multiplier \(X_j\) par une constante \(c \implies\) les coefficients OLS sont divisés par \(c\); peu importe comment la \(j\)\ieme{} variable est scalée, \(X_j\betaEstimated{j}\) est toujours le même). En revanche, les coefficients Ridge peuvent changer drastiquement (à cause de la somme des coefficients au carré).

            \paragraph{Biais et variance}
                Posons \(R = X^T X\)
                \begin{align*}
                    \betaRidge &= (X^T X + \lambda \identity_p)^{-1} X^T y = (R + \lambda \identity_p)^{-1} R(R^{-1} X^T y)\\
                    &= (R + \lambda \identity_p)^{-1} R \betaLS = \left[R(\identity_p + \lambda R^{-1})\right]^1 R \betaLS\\
                    &= (\identity_p + \lambda R^{-1})^{-1} \betaLS
                \end{align*}
                
                Posons \(W_\lambda = (\identity_p + \lambda R^{-1})^{-1}\)
                \begin{align*}
                    \expectation[\betaRidge] &= \expectation[W_\lambda \betaLS] = W_\lambda B \not= \beta & \text{Si \(\lambda \not= 0\)}\\
                    \variance[\betaRidge] &= \variance[W_\lambda \betaLS] = W_\lambda \variance[\betaLS] W_\lambda^T \leq \variance[\betaLS]
                \end{align*}
                On a donc un biais mais un plus petite variance.

            \paragraph{Décomposition en valeurs singulières}
                On peut utiliser la décomposition en valeurs singulières\index{SVD}, c'est-à-dire \(X = U D V^T\). Donc,
                \begin{align*}
                    X^T X &= (UDV^T)^T UDV^T\\
                    &= VD^TU^T UDV^T\\
                    &= VD^T DV^T & \text{Car les colonnes de \(U\) sont orthonormales}\\
                    &= V D^2 V^T & \text{Car \(D\) est diagonale}\\
                    \\
                    \betaLS &= (X^T X)^{-1} X^T y & \text{Définition}\\
                    &= V D^{-1} U^T y & X = U D V^T \text{ et \(V^{-1} = V^T\) (car colonnes orthonormales)}\\
                    &= V D^{-2} D U^T y\\
                    \betaRidge &= (X^T X + \lambda \identity_p)^{-1} X^T y & \text{Définition}\\
                    &= (V D^2 V^T + \lambda VV^T)^{-1} VDU^T y & VV^T = \identity_p \text{ car colonnes orthonormales}\\
                    &= (V(D^2 + \lambda \identity_p) V^T)^{-1} V D U^T y\\
                    &= V(D^2 + \lambda \identity_p)^{-1} V^T V D U^T y\\
                    &= V(D^2 + \lambda \identity_p)^{-1} D U^T y
                \end{align*}
                Les deux expressions ont quasiment la même formulation à une différence près : \(+ \lambda\) pour Ridge. On peut aussi déterminer l'expression des \(\estimation{y}\) :

                \begin{align*}
                    \yLS &= X \betaLS\\
                    &= U D V^T V D^{-2} D U^T y\\
                    &= U D D^{-2} D U^T y\\
                    &= U U^T y\\
                    \yRidge &= X\betaRidge\\
                    &= X V (D^2 + \lambda \identity_p)^{-1} D U^T y\\
                    &= U D V^T V (D^2 + \lambda \identity_p)^{-1} D U^T y\\
                    &= U D (D^2 + \lambda \identity_p)^{-1} D U^T y\\
                    &= U \diagonal(\frac{d_j^2}{d_j^2 + \lambda}) U^T y & \text{\(D\) est diagonale}
                \end{align*}

                Comme \(\lambda \geq 0\), on a \(\frac{d_j^2}{d_j^2 + \lambda} \leq 1\). On a donc bien un effet de shrinkage. Quand \(d_j^2\) est petit, le shrinkage est plus important. La variable \(z_j = X v_j = u_j d_j\) est la \(j\)\ieme{} PC de \(X\).

        \subsubsection{Lasso}
            \begin{definition}
                La \textit{régression Lasso}\index{Lasso}\nomenclature{Lasso}{Least absolute shrinkage and selection operator} est le problème d'optimisation :
                \begin{align*}
                    &\min_\beta \left\{\sum_{i=1}^n \left(y_i - \beta_0 - \sum_{j=1}^p \beta_j x_{ij}\right)^2\right\}\\
                    &\text{sous contrainte} \sum_{j=1}^p |\beta_j| \leq s
                \end{align*}
                avec \(s \geq 0\) un hyper-paramètre.

                Notons \(\betaLasso\) l'estimation des coefficients pour la régression Lasso.

                Si \(s = 0, \betaLasso = (0, \dots, 0)\). Si \(s = \infty, \betaLasso = \betaLS\). Si \(s \in ]0, \infty[\), on fait du tradeoff.
            \end{definition}

            \begin{definition}
                De manière équivalente, on peut exprimer la régression Lasso comme :
                \[
                    \min_\beta \sum_{i=1}^n \left(y_i - \beta_0 - \sum_{j=1}^p \beta_j x_{ij}\right)^2 + \lambda \sum_{j=1}^p |\beta_j|
                \]
                où \(\lambda \geq 0\) est un hyper-paramètre.

                Si \(s = 0, \betaLasso = \betaLasso\). Si \(s = \infty, \betaLasso = (0, \dots, 0)\). Si \(s \in ]0, \infty[\), on fait du tradeoff.
            \end{definition}

            \paragraph{Un cas simple}
                Supposons que \(n = p\) et que \(X = \identity_p = \identity_n\). Dans ce cas, la régression Lasso est 
                \[
                     \min_\beta \sum_{j=1}^p (y_j - \beta_j)^2 + \lambda \sum_{j=1}^p |\beta_j|
                \]
                Alors, on a :
                \[
                    \betaLasso_j = \begin{cases}
                        y_j - \frac{\lambda}{2} & \text{si } y_j > \frac{\lambda}{2}\\
                        y_j + \frac{\lambda}{2} & \text{si } y_j < \frac{-\lambda}{2}\\
                        0                       & \text{si } |y_j| \leq \frac{\lambda}{2}
                    \end{cases}
                \]

                La régression Lasso shrink les coefficients OLS vers zéro par la même constante \(\frac{\lambda}{2}\) et entièrement vers zéro quand ces coefficients sont plus petits que \(\frac{\lambda}{2}\) en valeur absolue.
    \subsection{Classification}
    \begin{lstlisting}
        # Régression logistique
        glm(formule, data = data, family = binomial)
        # LDA
        lda(formule, data = data)
        # QDA
        qda(formule, data = data)
        # k-NN (attention, il faut avoir train.X, train.y et test.X)
        knn(train.X, test.X, train.y, k = 1)
        # Arbres
        tree(formule, data = data)
        # Forêts aléatoires en utilisant m variables et n arbres
        randomForest(formule, data = data, mtry = m, ntrees = n)
    \end{lstlisting}

    \subsubsection{Récupérer probabilités}
        \lstinline{predict} supporte le paramètre \lstinline{type = "prob"} pour récupérer les probabilités pour certains modèles. Pour d'autres (comme \lstinline{glm}), il faut \lstinline{type = "response"}.
    \subsection{Réduction de dimensionnalité (PCA)}
    \begin{lstlisting}
        # PCA en scalant et centrant les données
        PCA = prcomp(data, scale = TRUE, center = TRUE)
        # Appliquer la transformation sur de nouvelles données
        predict(PCA, newdata)
    \end{lstlisting}

    \clearpage
    \addcontentsline{toc}{section}{Index}
    \printindex{}
    \clearpage
    \printnomenclature

    \clearpage
    \printbibliography{}
\end{document}