\subsection{Classification}
    \begin{definition}
        Pour la \textit{classification}\index{Classification}, on a que la variable de sortie \(Y\) est discrète (ou qualitative) et \(Y \in \class\) où \(\class = \{\class_1, \dots, \class_K\}\) est l'ensemble des classes possibles.

        On veut construire un \textit{classificateur}\index{Classificateur} \(C(X)\) qui assigne une classe de \(\class\) à une observation \(x\). En faisant ça, le classificateur découpe l'espace en \textit{régions de décision}\index{Région de décision} \(\region_k\) tels que tous les points dans \(\region_k\) sont assignés à la classe \(\class_k\). On veut aussi pouvoir mesurer l'incertitude des classifications (les probabilités) et comprendre le rôle des différentes variables.
    \end{definition}
    
    \subsubsection{Classificateur optimal}
        On veut minimiser l'erreur de prédiction :
        \[
            \expectation_{Y,X}[L(Y, C(X))] = \expectation_X\left[\sum_{i=1}^K L(\class_k, C(X))\distribution(\class_k|X)\right]
        \]

        Comme pour la régression, on laisse tomber le \(\expectation_X\) :
        \[
            C^*(x) = \argmin_{c \in \class} \sum_{k=1}^K L(\class_k, c)P(\class_k|X = x)
        \]

        Supposons qu'on utilise la \textit{zero-one loss}\index{Zero-one loss}, c'est-à-dire \(L(y, \estimation{y}) = \identity(y \not= \estimation{y})\) :
        \begin{align*}
            \sum_{k=1}^K L(\class_k, c)\distribution(\class_k|X = x) &= \sum_{k:\class_k \not=} 1P(\class_k|X = x) + 0P(c | X = x) & \text{Par la fonction de perte}\\
            &= \sum_{k=1}^K P(\class_k | X = x) - P(c | X = x)\\
            &= 1 - P(c | X = x) & \text{Car somme des proba = 1}
        \end{align*}

        Donc :
        \[
            C^*(x) = \argmin_{c\in\class} (1 - P(c | X = x)) = \argmax_{c\in\class} P(c | X = x)
        \]

        En d'autres termes, \(C^*(x) = \class_k\) si \(P(\class_k | X = x) = \max_{c\in\class} P(c|X=x)\).

        \begin{definition}[Classificateur de Bayes]
            Les \textit{probabilités conditionnelles de classe}\index{Probabilités conditionnelles de classe} sont :
            \[
                p_k(x) = P(Y = \class_k | X = x)
            \]

            Le \textit{classificateur de Bayes}\index{Classificateur!de Bayes} pour une observation \(x\) est :
            \begin{align*}
                C(x) &= \class_j & \text{si } p_j(x) = \max\{p_1(x), p_2(x), \dots, p_K(x)\}
            \end{align*}
        \end{definition}

        Ce classificateur donne la plus petite erreur moyenne. Cependant, comme on ne connait pas \(p_k(x)\), on ne sait pas le construire en pratique.

        \begin{definition}
            Le \textit{Bayes error rate}\index{Bayes error rate} est la plus petite erreur possible pour un classificateur :
            \[
                1 - \expectation[\max_j \distribution(Y = \class_j | X)]
            \]
        \end{definition}

    \subsubsection{Erreurs}
        \begin{definition}
            Au lieu du MSE, on utilise l'\textit{error rate}\index{Error rate} (la fraction de données mal classées) :
            \[
                \frac{1}{n}\sum_{i=1}^n \identity(y_i \not= C(x_i))
            \]
        \end{definition}

    \subsubsection{k-Nearest Neighbours}
        \begin{definition}
            Le \textit{k-Nearest Neighbours}\index{k-Nearest Neighbours|see {k-NN}} (ou \textit{k-NN}\index{k-NN})\nomenclature{k-NN}{k-Nearest Neighbours} est une méthode de classification. Soit une observation \(x_0\) :
            \begin{itemize}
                \item Trouver les \(k\) points les plus proches de \(x_0\) dans l'ensemble d'entraînement. Notons-les \(\neighbours_0\).
                \item Estimer les probabilités conditionnelle \(P(Y = \class_j | X = x_0) = \frac{1}{k} \sum_{i\in\neighbours_0}\identity(y_i = \class_j)\).
                \item Classer \(x_0\) dans la classe avec la plus haute probabilité.
            \end{itemize}
        \end{definition}

    \subsubsection{Régression linéaire et logistique}
        \paragraph{Deux classes}
        Dans les cas où \(\class = \{\class_1, \class_2\}\), on peut utiliser la régression logistique en rajoutant une variable \(Y\) :
        \[
            Y = \begin{cases}
                0 & \text{si } \class_1\\
                1 & \text{si } \class_2
            \end{cases}
        \]

        On obtient un classificateur équivalent au LDA\index{LDA} et, intuitivement, la régression va donner les probabilités d'avoir \(P(Y = 1)\). Cependant, la régression ne garantit de donner des nombres entre 0 et 1. Il faut donc utiliser la \textit{régression logistique}\index{Régression!Logistique}. De plus, la régression logistique (multiclasse) est capable de gérer les cas où il y a plus que deux classes.

        \begin{definition}
            La \textit{régression logistique} (pour deux classes) est de la forme :
            \[
                p(X) = \frac{e^{\beta_0 + \beta_1 X_1 + \dots + \beta_pX_p}}{1 + e^{\beta_0 + \beta_1X_1 + \dots + \beta_pX_p}} = \frac{e^{\beta^TX}}{1 + e^{\beta^TX}} = \sigma(\beta^TX)
            \] 
            avec \[\sigma(s) = \frac{e^s}{1+e^s} = \frac{1}{1 + e^{-s}}\]

            On a que la transformation \textit{log odds}\index{Log odds} ou \textit{logit}\index{Logit} de \(p(X)\) est :
            \[
                \log\left(\frac{p(X)}{1 - p(X)}\right) = \beta_0 + \beta_1X
            \]
        \end{definition}

        Utilisons le MLE\index{MLE} pour trouver les valeurs des paramètres :
        \begin{align*}
            L(\beta, \{y_i, x_i\}_{i=1}^n) &= \prod_{i=1}^n p(y_i | X = x_i)\\
            &= \prod_{i:y_i = 1} p(Y = 1|X = x_i) \prod_{i:y_i=0} p(Y = 0 | X = x_i)\\
            &= \prod_{i:y_i = 1} p(x_i) \prod_{i:y_i = 0} (1 - p(x_i)) & \text{Par les probabilités}\\
            &= \prod_{i:y_i = 1} \sigma(\beta^T x_i) \prod_{i:y_i=0} (1 - \sigma(\beta^T x_i)) & \text{Par définition de } p(x_i)\\
            &= \prod_{i:y_i = 1} \sigma(\beta^T x_i) \prod_{i:y_i=0} (-\beta^T x_i)
        \end{align*}

        En passant au log :
        \begin{align*}
            l(\beta, \{y_i, x_i\}_{i=1}^n) &= \sum_{i:y_i=1} \log(\sigma(\beta^T x_i)) \sum_{i:y_i=0} \log(\sigma(-\beta^T x_i))\\
            &= \sum_{i=1}^n \left(y_i \log(\sigma(\beta^T x_i)) + (1 - y_i) \log(\sigma(-\beta^T x_i))\right) & \text{Pour avoir une seule somme}
        \end{align*}

        Maximiser ça équivaut à minimiser :
        \begin{align*}
            \frac{-1}{N} \sum_{i=1}^n \left(y_i \log(\sigma(\beta^T x_i)) + (1 - y_i)\log(\sigma(-\beta^T x_i)\right) = \frac{1}{N}\sum_{i=1}^n\left(y_i \log\left(\frac{1}{\sigma(\beta^T x_i)}\right) + (1 - y_i) \log\left(\frac{1}{\sigma(-\beta^T x_i)}\right)\right)
        \end{align*}

        Au lieu d'avoir \(y_i \in \{0, 1\}\), on va supposer que \(y_i \in \{-1, +1\}\) :
        \begin{align*}
            \frac{1}{N}\sum_{i=1}^n \left(\identity\{y_i = +1\}\log\left(\frac{1}{\sigma(\beta^T x_i)}\right) + \identity\{y_i=-1\} \log\left(\frac{1}{\sigma(-\beta^T x_i)}\right)\right) &= \frac{1}{N} \sum_{i=1}^n \left(\log\left(\frac{1}{\sigma(y_i \beta^T x_i)}\right)\right)\\
            &= \frac{1}{N} \sum_{i=1}^n \log\left(1 + e^{-y_i \beta^T x_i}\right)
        \end{align*}

        Donc, maximer MLE revient à minimiser la \textit{logistic loss}\index{Logistic loss} :
        \[
            L(y, s) = \log(1 + e^{-ys})
        \]
        avec \(s = \beta^T x\).

        Il existe d'autres fonctions de coût pour la classification binaire :
        \begin{itemize}
            \item \textit{zero-one loss}\index{Zero-one loss} : \(L(y, s) = L(ys) = \identity\{ys < 0\}\); pas utilisable en pratique car non-convexe et non-smooth;
            \item \textit{squared loss}\index{Squared loss} : \(L(y, s) = L(ys) = (1 - ys)^2\);
            \item \textit{hinge loss}\index{Hinge loss} : \(L(y, s) = L(ys) = \max\{0, 1 - ys\}\);
            \item \dots
        \end{itemize}

        \paragraph{Plus que deux classes}
        Si on a \(K\) classes, il suffit de créer \(K\) classifications binaires et de choisir la classe avec la plus haute probabilité \(P(Y = k | X\) :
        \[
            P(Y = k | X) = \frac{e^{\beta_{0k} + \beta_{1k}X_1 + \dots + \beta_{pk}X-p}}{\sum_{l=1}^K}
        \]

        On appelle ça \textit{multiclass logistic regression}\index{Multiclass Logistic Regression} ou \textit{Multinomial Regression}\index{Multinomial Regression}.

    \subsubsection{Linear Discriminant Analysis}
        \begin{definition}
            Pour la \textit{Discriminant Analysis}, on va modéliser la distribution de \(X\) dans chaque classe (\(P(X | Y\)) séparément et utiliser le théorème de Bayes pour obtenir \(P(Y | X)\).

            Si on utilise des distributions normales pour chaque classe, on obtient la \textit{Linear Discriminant Analysis}\index{Linear Discriminant Analysis|see {LDA}} (noté \textit{LDA}\index{LDA}) ou \textit{quadratic}.
        \end{definition}

        \begin{definition}[Théorème de Bayes]
            Le \textit{théorème de Bayes}\index{Théorème de Bayes} appliqué à la classification :
            \[
                P(Y = k | X = x) = \frac{P(X = x | Y = k)P(Y = k)}{P(X = x)}
            \]
        \end{definition}

        Pour la discriminant analysis, on écrit :
        \[
            P(Y = k | X = x) = \frac{\pi_k f_k(x)}{\sum_{l=1}^K \pi_l f_l(x)}
        \]
        avec \(f_k(x) = P(X = x | Y = k)\) la \textit{densité} pour \(X\) dans la classe \(k\) et \(\pi_k = P(Y = k)\) la \textit{probabilité a priori} pour la classe \(k\).

        Ici, on suppose que la densité suit une loi normale (une par classe).

        Pour classer un point, on regarde quelle est la classe avec la plus haute densité pour ce point.

        Avantages sur la régression logistique :
        \begin{itemize}
            \item Quand les classes sont bien séparées, la régression logistique est instable pour les valeurs des paramètres. LDA ne subit pas ça.
            \item Si on a peu de données et que la distribution des variables est approximativement normale dans chaque classe, LDA est plus stable que la régression logistique.
            \item LDA est plus sympa à utiliser quand on a plus que deux classes (parce que `provides low-dimensional views of the data').
        \end{itemize}

        \paragraph{Une seule variable}
            La densité gaussienne a, dans le cas où \(p = 1\), la forme :
            \[
                f_k(x) = \frac{1}{\sqrt{2\pi}\sigma_k} e^{-\frac{1}{2}\left(\frac{x - \mu_k}{\sigma_k}\right)^2}
            \]
            
            Supposons que \(\forall k, \sigma_k = \sigma\) (même variance quelque soit la classe). On obtient, via le théorème de Bayes :
            \[
                p_k(x) = \frac{\frac{\pi_k}{\sqrt{2\pi}\sigma} e^{-\frac{1}{2}\left(\frac{x - \mu_k}{\sigma}\right)^2}}{\sum_{l=1}^K \frac{\pi_l}{\sqrt{2\pi}\sigma} e^{-\frac{1}{2}\left(\frac{x - \mu_l}{\sigma}\right)^2}}
            \]

            Pour trouver la classe, il faut trouver le \(k\) qui maximise \(p_k(x)\) :
            \[
                \max_k p_k(x) \equiv \max_k \log(p_k(x)) \equiv \max_k \log\left(\frac{\pi_k}{\sqrt{2\pi}\sigma} e^{-\frac{1}{2\sigma^2}(x - \mu_k)^2}\right) \equiv \max_k \delta_k(x)
            \]
            avec \[\delta_k(x) = \frac{x \mu_k}{\sigma^2} - \frac{\mu_k^2}{2\sigma^2} + \log(\pi_k)\]

            Les \textit{fonctions discriminantes}\index{Fonctions discriminantes} \(\delta_k(x)\) sont des fonctions linéaires en \(x\). La frontière de décision entre chaque paire de classes \(k\) et \(l\) est décrite par \(\{x : \delta_k(x) = \delta_l(x)\}\).

            \begin{theorem}
                Si \(K = 2\) et \(\pi_1 = \pi_2 = 0.5\), alors la frontière de décision est à \(x = \frac{\mu_1 + \mu_2}{2}\).
            \end{theorem}
            \begin{proof}
                Il faut avoir :
                \begin{align*}
                    \delta_1(x) &= \delta_2(x)\\
                    \Rightarrow \frac{x\mu_1}{\sigma^2} - \frac{\mu_1^2}{2\sigma^2} &= \frac{x\mu_2}{\sigma^2} - \frac{\mu_2^2}{2\sigma^2} & \text{Car } \log(\pi_1) = \log(\pi_2) \text{ car } \pi_1 = \pi_2\\
                    \Rightarrow 2x(\mu_1 - \mu_2) &= \mu_1^2 - \mu_2^2\\
                    \Rightarrow x &= \frac{\mu_1 + \mu_2}{2}
                \end{align*}
            \end{proof}

            Il reste à être capable d'estimer les paramètres :
            \begin{align*}
                \estimation{\pi}_k &= \frac{n_k}{n}\\
                \estimation{\mu}_k &= \frac{1}{n_k} \sum_{i:y_i = k} x_i\\
                \estimation{\sigma}^2 &= \frac{1}{n - K} \sum_{k=1}^K \sum_{i:y_i = k} (x_i - \estimation{\mu}_k)^2 = \sum_{k=1}^K \frac{n_k - 1}{n - K} \estimation{\sigma}_k^2
            \end{align*}
            avec \(\estimation{\sigma}_k^2 = \frac{1}{n_k - 1}\sum_{i:y_i = k} (x_i - \estimation{\mu}_k)^2\).

        \paragraph{Plus qu'une variable}
            Dans le cas où \(p > 1\), la fonction de densité est de la forme :
            \[
                f(x) = \frac{1}{\sqrt{(2\pi)^p} |\Sigma|^{1/2}} e^{-\frac{1}{2}(x - \mu)^T \Sigma^{-1}(x-\mu)}
            \]

            Les fonctions discriminantes sont :
            \[
                \delta_k(x) = x^T \Sigma^{-1} \mu_k - \frac{1}{2}\mu_k^T \Sigma^{-1} \mu_k + \log(\pi_k)
            \]
            \(\delta_k(x)\) est toujours une fonction linéaire en \(x\).

        \paragraph{Autres formes de Discriminant Analysis}
            \begin{itemize}
                \item Avec des gaussiennes mais différents \(\Sigma_k\) pour chaque classe, on obtient \textit{Quadratic Discriminant Analysis}\index{Quadratic Discriminant Analysis}.
                \item Avec \(f_k(x) = \prod_{j=1}^p f_{jk}(x_j)\) pour chaque classe, on obtient le classificateur \textit{Naive Bayes}\index{Naive Bayes}. Pour les gaussiennes, ceci équivaut à avoir des \(\Sigma_k\) diagonales.
                \item En changeant le \(f_k(x)\), on peut donc avoir plein d'autres formes, y compris des non-paramétriques.
            \end{itemize}

        \paragraph{Logistic Regression vs LDA}
            Pour un problème à deux classes, on peut montrer que pour LDA
            \[
                \log\left(\frac{p_1(x)}{1 - p_1(x)}\right) = \log\left(\frac{p_1(x)}{p_2(x)}\right) = c_0 + c_1 x_1 + \dots + c_p x_p
            \]

            Donc, dans ce cas, LDA a la même forme que la régression logistique. La différence vient de comment les paramètres sont estimés :
            \begin{itemize}
                \item La régression logistique utilise la conditional likelihood basée sur \(P(Y | X)\) (cette méthode est appelée \textit{discriminative learning}\index{Discriminative Learning});
                \item LDA utilise la likelihood complète basée sur \(P(X, Y)\) (appelée \textit{generative learning}\index{Generative Learning}).
            \end{itemize}
            Malgré ça, les résultats sont souvent similaires.

    \subsubsection{Naive Bayes}
        \begin{definition}
            Pour le \textit{Naive Bayes}\index{Naive Bayes}, on suppose que les variables sont indépendantes dans chaque classe. Ceci est pratique quand \(p\) est très grand.

            Le \textit{Gaussian naive Bayes} suppose que chaque \(\Sigma^k\) est diagonale :
            \[
                \delta_k(x) \propto \log\left(\pi_k \prod_{j=1}^p f_{kj}(x_j)\right) = - \frac{1}{2} \sum_{j=1}^p \frac{(x_j - \mu{kj})^2}{\sigma_{kj}^2} + \log(\pi_k)
            \]
            et peut être utilisé pour mixer des vecteurs de variables (qualitatives et quantitatives). Si \(X_j\) est qualitative, il suffit de remplacer \(f_{kj}(x_j)\) par une function de probabilité de masse (PMF).
        \end{definition}